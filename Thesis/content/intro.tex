\chapter{Einführung}\label{ch:intro}

Das Smartphone ist heutzugtage der stete Begleiter eines Menschen. \enquote{Zwei Drittel der Bevölkerung und nahezu jeder 14- bis 29-Jährige geht darüber ins Netz.} \cite{usage} Auch die Prognose zeigt, das der Absatzmarkt immer weiter steigen wird (Abbildung \ref{fig:prognose_fig}).

\begin{figure}[H]
	\begin{center}
		\includegraphics[width=0.86\textwidth]{images/prognose-zur-anzahl-der-smartphone-nutzer-weltweit-bis-2020.png}
		\caption{Prognose zur Anzahl der Smartphone-Nutzer weltweit von 2012 bis 2020 (in Milliarden) \cite{prognose}}
		\label{fig:prognose_fig}
	\end{center}
\end{figure}

Umso wichtiger ist es das die Softwareentwicklung diesen Trend ernst nimmt. Der ehemalige Google-Chef Eric Schmidt sagte bereits 2010: \enquote{Googles Devise heißt jetzt \enquote{Mobile first}}. 
Diese Devise wird auch heute noch von vielen Unternehmen verfolgt, das ist der Grund weswegen in den einzelnen Stores heutzutage so viele Apps angeboten werden. Bei Android im Playstore sind es im Oktober 2016 ca 2.432.000 Apps \cite{play_store}, bei Apple im App Store sind es ca 2.000.000 Apps (Stand Juni 2016) \cite{app_store}. Neben Googles Android und Apples IOs gibt es noch andere Betriebssysteme, wie Microsofts Windows Phone oder Blackberrys Blackberrys OS und noch ein paar andere. Jedoch bestimmen die beiden erstgenannten Systeme den Markt (Abbildung \ref{fig:os_fig}).

\begin{figure}[H]
	\begin{center}
		\includegraphics[width=0.86\textwidth]{images/os.jpg}
		\caption{Der weltweite Marktanteil von Smartphone-Betriebssysteme. \cite{os}}
		\label{fig:os_fig}
	\end{center}
\end{figure}

Ein großes Problem in dieser Branche ist die ernorme Schnelllebigkeit. Vor allem im Bereich Android werden beinahe wöchentlich neue Geräte durch unterschiedliche Hersteller vorgestellt. Neben den hardware-spezifischen Unterschieden wie: Größe und Auflösung des Displays, Größe des verbauten RAMs, Prozessorleistung und so weiter, muss ein Entwickler für Android Applikations zusätzlich noch mit den hersteller-spezifischen Eigenentwicklungen von Android kämpfen. So nutzt zum Beispiel HTC, ihre HTC Sense \cite{htc_sense}, Samsung setzt auf TouchWiz \cite{touchwiz}. Neben diesen herstellergebundenen Systeme gibt es zusätzlich noch Custom-ROMs, welche durch den Nutzer selbst installiert werden können. Die beliebtesten ROMs sind Paranoid Android und CyanogenMod \cite{rom}. Jedes dieser Systeme besitzt auserdem noch unterschiedliche Versionen so befindet sich Vanilla Android im Moment in der Version 7.0 Nougat.
Durch die einzelnen Systeme und deren Versionen entstehen ernorm viele Anforderungen an die Software, welche möglichst eine Vielzahl der Varianten unterstützen sollte. Um diese Anforderungen stemmen zu können, muss ein extrem hoher Anteil an Wartung in die Entwicklung und Instandhaltung einfließen. Zusätzlich zu dem Mehraufwand muss sich der Android-Entwickler ständig über die neuen Spezifikationen informieren.
\section{Motivation}\label{sec:motivation}
Meine Motivation
\section{Zielsetzung}\label{sec:target}
Zielsetzung
\section{Aufbau der Arbeit}\label{sec:structure}
Aufbau
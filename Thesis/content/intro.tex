\chapter{Einführung}\label{ch:intro}

Das Smartphone ist heutzugtage der stete Begleiter eines Menschen. \enquote{Zwei Drittel der Bevölkerung und nahezu jeder 14- bis 29-Jährige geht darüber ins Netz.} \cite{usage} Auch die Prognose zeigt, das der Absatzmarkt immer weiter steigen wird (Abbildung \ref{fig:prognose_fig}).

\begin{figure}[H]
	\begin{center}
		\includegraphics[width=0.86\textwidth]{images/prognose-zur-anzahl-der-smartphone-nutzer-weltweit-bis-2020.png}
		\caption{Prognose zur Anzahl der Smartphone-Nutzer weltweit von 2012 bis 2020 (in Milliarden) \cite{prognose}.}
		\label{fig:prognose_fig}
	\end{center}
\end{figure}

Umso wichtiger ist es das die Softwareentwicklung diesen Trend ernst nimmt. Der ehemalige Google-Chef Eric Schmidt sagte bereits 2010: \enquote{Googles Devise heißt jetzt \enquote{Mobile first}}. 
Diese Devise wird von vielen Unternehmen verfolgt, das ist der Grund weswegen in den einzelnen Stores heutzutage so viele Apps angeboten werden. Bei Android im Playstore sind es im Oktober 2016 ca. 2,4 Millionen Apps \cite{play_store}, bei Apple im App Store sind es ca. 2 Millionen Apps (Stand Juni 2016) \cite{app_store}. Neben Googles Android und Apples IOs gibt es noch andere Betriebssysteme, beispielsweise Microsofts Windows Phone oder Blackberrys Blackberrys OS. Jedoch bestimmen die beiden erstgenannten Systeme den Markt (Abbildung \ref{fig:os_fig}).

\begin{figure}[H]
	\begin{center}
		\includegraphics[width=0.86\textwidth]{images/os.jpg}
		\caption{Der weltweite Marktanteil von Smartphone-Betriebssysteme. \cite{os}}
		\label{fig:os_fig}
	\end{center}
\end{figure}

Jede dieser Applikationen wurden einzeln für sich entwickelt und implementiert. Bei jedem Update zum Beispiel des Systems, müssen alle Anwendungen gewartet und überarbeitet werden, um die volle Funktionalität zu gewährleisten.

Würden einige Applikationen jedoch genauer analysiert werden, wäre das Ergbniss, dass in jeder dieser Anwendungen Codepassagen vorhanden sind, welche einen ähnlichen beziehungsweise den selben Zweck erfüllen. Werden diese Stellen im Programmcode abstrahiert, gibt es die Möglichkeit diese generieren zu lassen. Um Code generieren zu lassen, benötigt man so genannte Code-Generatoren. 

Im Bereich der Backend-Entwicklung gibt es bereits verschiedene Projekte die sich damit befassen. Ein Beispiel wäre der \enquote{CRUD Admin Generator} \cite{generators}. Die Hochschule für angewandte Wissenschaften Würzburg-Schweinfurt entwickelt  unter der Leitung von Prof. Dr. Peter Braun auch einen Code-Generator unter dem Namen: \ac{gemara}. Mit Hilfe solcher Generatoren für den Bereich von Mobilen Applikationen, könnte der Entwicklungs- und Wartungsaufwand reduziert werden. 

Führt ein Systemupdate dazu, dass die Implementierung von verschiedenen Anforderungen nicht weiter funktionsfähig ist, muss dies nur einmalig an der entsprechenden Stelle im Code-Generator geändert werden und nicht in jeder Applikation einzeln. 

\section{Motivation}\label{sec:motivation}
Im Rahmen des Projektes \ac{gemara} gab es bereits Arbeiten, welche sich mit dem Thema der Generierung von Android Aktivities beschäftigt. Die dabei entstandenen Lösungen, reslultierten darin, das dass generieren von Aktivities zu Problemen führt. Deshalb beschäftigt sich diese Ausarbeitung damit, nicht eine komplette Aktivity zu erzeugen, sondern sogenannte Komponenten.

Eine Komponente, ist im wesentlichen eine kleine Anwendung für sich, welche nur eine einzige Aufgabe erfüllt. Dies könnte zum Beispiel das Anzeigen eines Dozenten in einer Campus-Applikation sein.

Aus den erzeugten Komponenten, kann eine Art \enquote{Bausatz} entstehen. Mit dessen Hilfe der Entwickler seine Applikation zusammen bauen kann. Dabei wird ihm freie Wahl gelassen, wie der Aufbau seiner Anwendung aussieht, er bedient sich nur an gegebener Stelle an den Komponenten. Dadurch reduziert sich der Entwicklungsaufwand für ihn.

Bewegen wir uns in der Domaine einer Hochschule, so kann eine Bibliotek mit den erzeugten Komponenten allen Studierenden zur Verfügung gestellt werden. Dadurch wäre jeder dieser Studierenden in der Lage eine persönliche Campus-Applikation zu entwickeln. Durch die einzelnen Komponenten kann dann sichergestellt werden, dass grundsätzliche Funktionalität bereits gewährleistet ist.

\section{Zielsetzung}\label{sec:target}
Das Ziel dieser Ausarbeitung liegt darin, das der Leser ein grundsätzliches Verständnis in der Implementierung von Custom Views in Android, Erstellung von Bibliotheken und der Datenkommunikation mittels einer REST-API erlangen. Mit diesem Wissen, sollte der Leser dann in der Lage sein Generatoren für Custom Views zu entwickeln, welche die oben genannten Grundfunktionen einer Applikation abdecken.

Darüber hinaus soll auch ein Gefühl dafür entwickelt werden, wann es Sinn macht eine View als Custom View zu implementieren und wann der benötigte Mehraufwand nicht mehr in Relation zu einem vernünftigen Ergebniss steht.


\section{Aufbau der Arbeit}\label{sec:structure}
Aufbau
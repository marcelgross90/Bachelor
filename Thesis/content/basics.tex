\chapter{Grundlagen}\label{ch:basics}
\section{\textit{\acf{rest}}}\label{sec:rest}
In dem generierten Projekt, sollen alle benötigten Daten mittels \textit{\ac{rest}} von dem zugehörigen, generierten \textit{Backend} geladen werden. 

 \textit{\ac{rest}} \cite{rest_fielding} ist ein Programmierparadigma, welches sich auf folgende Prinzipien stützt: \textit{Client-Server}, \textit{Zustandslose Kommunikation}, \textit{Caching}, \textit{Uniform Inferface}, \textit{Layered System} und dem optionalen Prinzip \textit{Code-on-Demand}.
Diese Arbeit berücksichtigt vor allem \textit{\acf{hateoas}} welches unter das Prinzip \textit{Uniform Interface} fällt. Es beschreibt, wie mit Hilfe eines \textit{endlichen Automaten} eine \textit{\ac{rest}-Architektur} entworfen werden kann.
Der Architekt einer \textit{\ac{rest}}-konformen \textit{\acf{api}} überlegt sich im voraus, wie der Applikation-Fluss in der späteren Anwendung aussehen soll. Dafür definiert er verschiedene \textit{States} und welche \textit{Transitionen} zum nächsten \textit{State} führen.

Als ein \textit{State} kann beispielsweise das Anzeigen aller Lecturer in einer Campus-Applikation angesehen werden.
Die \textit{Transition} hingegen ist zum Beispiel ein \textit{Link} im \textit{Link-Header} der Antwort, oder ein Attribut, der empfangen Ressource. 

Wird das \textit{\ac{api}} mit Hilfe eines \textit{endlichen Automaten} entwickelt, kann diese dem \textit{Client}-Entwickler als Anleitung zum erstellen des \textit{Clients} dienen. Er benötigt diesbezüglich einen \acf{url}, welcher auf den initialen \textit{State} des \textit{endlichen Automaten} führt. Dieser liefert dann alle, zu diesem Zeitpunkt möglichen, \textit{Transitionen} zurück. Mit Hilfe dieser \textit{Transitionen}, kann sich der Entwickler dann zum nächsten \textit{State} bewegen. Auch dieser \textit{State} liefert neben den Ressourcen, alle möglichen weiteren \textit{Transitionen} zurück. 
Wenn der Entwickler sich so durch die \textit{States} bewegt, bekommt er die benötigten Informationen zum Aufbau und Ablauf der Applikation.

Die Abbildung \ref{fig:finite_state_machine} zeigt einen solchen Automaten. Der Einstiegspunkt ist der \textit{State} \enquote{Dispatcher} dieser liefert die \textit{Transition} zum \textit{State} \enquote{Collection} zurück. Dieser \textit{State}, verfügt über alle Informationen die benötigt werden um eine Collection der betroffenen Ressource anzuzeigen, weiterhin verfügt er auch das Wissen, über die beiden nächsten \textit{Transitionen} zu den \textit{State}s \enquote{Create} und \enquote{Single}. Wie der Name des \textit{States} annehmen lässt, wird der \textit{State} \enquote{Create} benötigt um eine neue Ressource anzulegen. Von diesem \textit{State} aus kann die Anwendung nur zurück zum \textit{State} \enquote{Collection}. Der \textit{State} \enquote{Single} enthält alle benötigten Daten um eine einzelne Ressource anzuzeigen. Vom hier kann die Anwendung zum \textit{State} \enquote{Update} oder \enquote{Delete} wechseln. Der \textit{State} \enquote{Update} ermöglicht es die Ressource zu bearbeiten. Von hier kann der Nutzer der Anwendung nur zum \textit{State} \enquote{Single} zurückkehren. Der \textit{State} \enquote{Delete} löscht die aktuelle Ressource und liefert die \textit{Transition} zum \textit{State} \enquote{Collection} zurück. 
Dieses Beispiel verdeutlicht noch einmal bildlich, das der Entwickler nur den Einstiegspunkt \enquote{Dispatcher} kennen muss. Die Anwendung liefert selbst alle benötigten Informationen um die Daten für die Anwendung nachzuladen.

\begin{figure}[H] \label{fig:finite_state_machine}
	\begin{center}
		\begin{tikzpicture}[->,>=stealth',shorten >=1pt,auto,node distance=3cm,
		semithick]
		\tikzstyle{every state}=[fill=none,text=black]
		
		\node[initial,state] (A) [minimum width=3cm]       {Dispatcher};
		\node[state]     (B) [right of=A, minimum width=2cm] {Collection};
		\node[state]     (D) [below of=B, minimum width=2cm] {Create};
		\node[state]     (C) [right of=B, minimum width=2cm] {Single};
		\node[state]     (E) [above of=C, minimum width=2cm] {Delete};
		\node[state]		 (F) [below of=C, minimum width=2cm] {Update};
		
		\path 
		(A) edge       (B)
		(B) edge [bend left] (C)
			edge [bend left] (D)
		(C)	edge 			 (E)
			edge [bend left] (B)
			edge [bend left] (F)
		(D) edge [bend left] (B)
		(E) edge  			 (B)
		(F) edge [bend left] (C);
		
		\end{tikzpicture}
			\caption[Aufbau eines \acl{rest}-\acl{api} mit Hilfe eines \textit{endlichen Automaten}.]{Aufbau eines \textit{\ac{rest}-\ac{api}} mit Hilfe eines \textit{endlichen Automaten}.}
		\label{fig:finite_state_machine}
	\end{center}
\end{figure}

\section{Entwicklung von Android \textit{CustomViews}}\label{sec:custom_view}
Die Software-Plattform Android basiert auf Linux und wird als Betriebssystem für mobile Endgeräte verwendet.
Das System wird als \textit{Open Source} Projekt von der Open Handset Alliance entwickelt \cite{open_handset_alliance}. Dabei ist ein Ziel, die Schaffung eines offenen Standards für mobile Endgeräte.
Die Entwicklung ist nicht abgeschlossen, die aktuelle Version ist 7.0 Nougat (Stand Feb. 2017).

Programme für diese Plattform nennt man Applikation oder kurz App. Eine App stellt alle nötigen Sourcen bereit, zum Beispiel den Programmcode, \textit{Layout} und Grafiken, die benötigt werden, um diese App auf einem Android-Endgerät auszuführen.
Mit Hilfe von \textit{Widgets} und \textit{Layouts} können \textit{Views} definiert werden. Diese \textit{Views} stellen dann die gewünschte Information auf dem Display dar. Die bekanntesten \textit{Widgets} sind: \textit{TextView}, \textit{Button} und \textit{EditText}. Die Anordnung dieser \textit{Widgets} erfolgt dann mit einem \textit{Layout}. Es gibt hierbei verschiendene \textit{Layouts} zur Auswahl. Beispielsweise das \textit{LinearLayout}, mit horizontaler oder vertikaler Orientierung. Ein weiteres Beispiel ist das \textit{RelativeLayout}.

Reichen die Standart-\textit{Layouts} und -\textit{Widgets} nicht aus, gibt es noch die Möglichkeit eigene zu entwickeln. Dies ermöglicht diese \textit{Views} um Attribute und Methoden zu erweitern. Diese können dann sowohl in der \textit{Layout-XML} als auch im Programm-Code angesprochen werden.

Ausgehend davon das eine Applikation eine Liste von Personen mit Hilfe einer \textit{RecyclerView} anzeigen soll, gibt es die Möglichkeit eine \textit{CardView} zu erzeugen, welche eine einzelne Person darstellt. Diese \textit{CardView} kann in einem \textit{XML-Layout} wie allgemein bekannt definiert werden. Um die \textit{View} dann mit den entsprechenden Informationen zu befüllen werden im \textit{Adapter} der \textit{RecyclerView} dann die einzelnen Attribute einzeln angesprochen und mit den erforderlichen Details befüllt.
Alternativ besteht die Möglichkeit eine \textit{CustomView} zu erzeugen, in diesem Fall eine \textit{PersonCardView}.
Hierfür sind folgende Schritte notwendig: Registrieren der \textit{CustomViews}, Definieren des Aufbaus der \textit{CustonView} erzeugen einer \textit{CustomView} Klasse.

\subsection{Registrieren der \textit{CustomView}}
Zur Erzeugung und Registrierung von \textit{CustomViews} wird eine Datei \enquote{attrs.xml} benötigt. Diese wird liegt im Ordner \enquote{values} im Verzeichnis \enquote{res}. 
In dieser \textit{XML}-Datei werden im \enquote{resoruces}-Bereich die einzelnen \textit{CustomViews} aufgelistet. Es besteht die Möglichkeit diesen \textit{Views} zusätzlich Attribute zuzuweisen. Ein Attribut besteht dabei immer aus einem Namen und einem Format. Dieses Format definiert den erwarteten Eingabewert. Es gibt folgende definierte Formate: \textit{string}, \textit{integer}, \textit{boolean} oder \textit{color}. 
Formate können kombiniert werden. Beispielsweise das Attribut \enquote{backgroundColor} könnte so definiert werden \textit{format=\enquote{color|string}}. Listing \ref{lst:attrs} zeigt den Aufbau einer \enquote{attrs.xml}-Datei.

\newpage
\begin{lstlisting}[label=lst:attrs,
language=xml,
firstnumber=1,
caption=Aufbau einer \textit{attrs.xml} - Datei]				  
<resources>
	<declare-styleable name="AttributeInput">
		<attr name="hintText" format="integer"/>
		<attr name="inputType" format="string"/>
	</declare-styleable>
	
	<declare-styleable name="PersonCardView" />
</resources>
\end{lstlisting}

\subsection{Definieren des Aufbaus der \textit{CustomView}}

Da die \textit{PersonCardView} eine \textit{CustomView} ist, welche aus verschiedenen \textit{Widgets} zusammengesetzt wurde, müssen diese auch definiert werden. Dies geschieht wie gewohnt mit Hilfe einer \textit{XML-Datei}, mit einer Ausnahme. \textit{Die Root-View} ist in diesem Fall keine \textit{CardView} sondern ein beliebiges anderes \textit{Layout}. Da die \textit{PersonCardView} von \textit{CardView} erbt und somit bereits eine \textit{CardView} ist.

\begin{lstlisting}[label=lst:personCardViewXml,
language=xml,
firstnumber=1,
caption=Aufbau der \textit{PersonCardView} mit Hilfe einer \textit{XML}-Datei]				  
<RelativeLayout xmlns:android="http://schemas.android.com/apk/res/android"
	android:id="@+id/relativeLayout"
	android:layout_width="match_parent"
	android:layout_height="wrap_content">

	<TextView
		android:id="@+id/first_name"
		android:layout_width="wrap_content"
		android:layout_height="wrap_content"
		android:text="@string/firstName"/>

	<TextView
		android:id="@+id/last_name"
		android:layout_width="wrap_content"
		android:layout_height="wrap_content"
		android:text="@string/last_name"/>

...

</RelativeLayout>
\end{lstlisting}

\subsection{Erzeugen einer \textit{CustomView} Klasse}

Hierfür wird eine neue Java-Klasse erzeugt, welche von \textit{CardView} erbt. Es kann auch direkt von der \textit{View}-Klasse geerbt werden und anschließend mithilfe der Methode \textit{onDraw}, welche überschrieben werden muss, den gewünschten Inhalt anzuzeigen Sei es nun Text, Formen oder Benutzereingaben.
In diesem Fall entspricht die \textit{CardView} weitestgehend bereits den Anforderungen, so das diese genutzt wird.
Die Vererbungsstruktur bringt mit sich, das die \textit{Konstruktoren} der \textit{CardView} implementiert werden müssen.
Die Anzahl dieser \textit{Konstruktoren} hängt von der Minimum \textit{SDK}-Versions des Projekts ab. Dieses Projekt nutzt das Minimum Level 12 somit müssen drei \textit{Konstruktoren} überschrieben werden. Ab einen Level von 21, sind es vier, da ein weiteres Attribut zur \textit{View} hinzugefügt wurde.

\begin{lstlisting}[label=lst:personCardView,
language=java,
firstnumber=1,
caption=\textit{Konstruktoren} der \textit{PersonCardView}]				  
public class PersonCardView extends CardView {

public PersonCardView(Context context) {
	super(context);
	init(context, null, 0);
}

public PersonCardView(Context context, AttributeSet attrs) {
	super(context, attrs);
	init(context, attrs, 0);
}

public PersonCardView(Context context, AttributeSet attrs, int defStyleAttr) {
	super(context, attrs, defStyleAttr);
	init(context, attrs, defStyleAttr);
}
...
}
\end{lstlisting}

Innerhalb der \textit{init}-Methode wird definiert, was die \textit{View} anzeigen beziehungsweise was sie tun soll. 
In diesem Beispiel werden die verwendeten \textit{Widgets} initialisiert. Besäße die \textit{PersonCardView} noch eigene Attribute, so würden diese im \textit{AttributeSet} übergeben und könnten daraus in ein \textit{TypedArray} geschrieben werden. Dieses \textit{TypedArray} muss am Ende \textit{recycled} werden, damit es für einen späteren Aufruf wieder zur Verfügung steht.

\newpage
Jetzt wird die \textit{PersonCardView} um eine Methode \textit{setPerson} erweitert. Diese ist angelehnt an die Methode \textit{setText} der \textit{TextView}. Sie ermöglicht das der \textit{PersonCardView} ein Objekt Person übergeben wird und füllt dann die entsprechenden \textit{Widgets} mit den dazugehörigen übergebenen Daten.

\begin{lstlisting}[label=lst:setPerson,
language=java,
firstnumber=1,
caption=\textit{setPerson} - Methode aus der \textit{PersonCardView}.]				  
public void setPerson(Person person) {
	this.firstName.setText(person.getFirstName());
	this.lastName.setText(person.getLastName());	
	...
}
\end{lstlisting}

Mit Hilfe dieser Methode wird die Nutztung der \textit{PersonCardView} vereinfacht. Im \textit{Adapter} der \textit{RecyclerView} wird jetzt nicht mehr jedes einzelne \textit{Widget} definiert und mit Informationen befüllt. Sondern nur noch die \textit{PersonCardView} und mit der \textit{setPerson} - Methode kann die komplette Karte mit den Daten einer Person mit nur einem Methodenaufruf befüllt werden.

\section{Software-Generatoren}\label{sec:generators}

Mit Software-Generatoren, ist es möglich Software generieren zu lassen. Dafür wird die Problemstellung der realen Welt so beschrieben, dass der Generator dies versteht, interpretieren und Programmcode erzeugen kann.

\subsection{\textit{Domänenspezifische Sprache (DSL)}}\label{sec:dsl}
Die Grundlage, um ein Model für einen Generator zu beschreiben ist die \textit{\acl{dsl}}.
Eine \textit{\acs{dsl}} ist eine Programmiersprache, welche auf die Probleme einer bestimmten Domäne ausgelegt ist \cite{dslHudak}. Dadurch dass diese Sprachen auf ein ganz bestimmtes Problem zugeschnitten sind, sind \textit{\acl{dsl}} in ihrer Ausdrucksfähigkeit beschränkter als herkömmliche Programmiersprachen wie beispielsweise Java, C++ oder C\# Eine \textit{\acl{dsl}} wird dafür entwickelt ein konkretes Problem so effizient wie möglich zu lösen, ohne die komplexen Strukturen des Programmcodes kennen zu müssen.
Bekannte \textit{\aclp{dsl}} sind: \textit{\ac{sql}}, \textit{Make} und \textit{\acf{html}}.

Die \textit{domänenspezifischen Sprachen} lassen sich in zwei Kategorien einteilen die \textit{internen} und die \textit{externen \acsp{dsl}}.

\subsubsection{\textit{Interne \acsp{dsl}}} \label{sec:intern}
Eine interne \textit{\acs{dsl}} wird auch \textit{embdded \acs{dsl}} genannt, weil sie keine eigene \textit{Syntax} und \textit{Grammatik} entwickelt. Sie bedienen sich der \textit{Hostsprache}. Das heißt sie nutzen die selbe Programmiersprache, in welcher auch das Resultat sein wird.
Jedoch wird die verwendete \textit{Hostsprache} eingeschränkt, so nutzt die \textit{\acl{dsl}} nur eine Teilmenge der Möglichkeiten \cite{dsl}. Die Nutzung von \textit{internen \acsp{dsl}} sind zum Beispiel: Es muss kein neuer \textit{Compiler} und \textit{Parser} geschrieben werden. Auch gibt es bereits integrierte Entwicklungsumgebungen (IDEs). Außerdem muss der Programmierer keine neue Sprache lernen, um die \textit{acl{dsl}} zu nutzen, sollte er die verwendete \textit{Hostsprache} bereits kennen. 

\subsubsection{\textit{Externe \acsp{dsl}}} \label{sec:extern}

Anders als die \textit{internen \acsp{dsl}} besitzen \textit{externen \acsp{dsl}} eine eigene Syntax. Dies macht die Entwicklung einer solchen \textit{\acl{dsl}} sehr viel aufwändiger, da nun ein eigener \textit{Parser} und \textit{Compiler} mitentwickelt werden muss \cite{dsl}. Jedoch bringt diese eigene \textit{Syntax} auch den Vorteil, dass die Sprache nicht auf die Besonderheiten einer \textit{Hostsprache} eingeschränkt ist. So können Anforderungen an die Domäne bereits beim schreiben des \textit{Compilers} mit validiert werden.

\subsection{\acf{gemara}}\label{sec:gemara}

Das Projekt \acs{gemara} beinhaltet ein Reihe von Software-Generatoren, deren Ziel es ist mobile und verteilte Applikationen basierent auf dem \textit{\acs{rest}} Architekturstil generieren zu lassen. Dafür wurde eine \textit{interne \acs{dsl}} entwickelt, mit dessen Hilfe sowohl \textit{Client}-seitige als auch \textit{Server}-seitige Applikationen beschrieben werden können.

Im Moment (Stand Februar 2017) ist es möglich ein \textit{WAR-Artefakt} für einen \textit{Tomcat-Webserver} generieren zu lassen. Dieses erzeugte Projekt kann auf eine \textit{relationale MYSQL-Datenbank} oder einer \textit{dokumentenbasierten CouchDB} zugreifen. Des weiteren ist ein Generator zur Erzeugung von Android-Applikationen sowie ein Generator für Polymer Webkomponenten in der Entwicklung.

\subsubsection{Aufbau von \acs{gemara}}

\acs{gemara} ist modular aufgebaut, jedes der einzelnen Module erfüllt einen eindeutig definierten Zweck.

\begin{figure}[H]
	\begin{center}
		\includegraphics[width=0.86\textwidth]{images/gemara.png}
		\caption{Aufbau von \acs{gemara}}
		\label{fig:gemara}
	\end{center}
\end{figure}

\begin{itemize}
	\item \textbf{Ratcliff} definiert ein Enfield-Model, mit Hilfe einer \textit{Fluent \acs{api}}.
	\item \textbf{Yeading} definiert eine Repräsentation eines Enfield-Models, mit \textit{\acf{yaml}} oder \textit{\acf{json}}, welches dann in nach Ratcliff übersetzt werden.
	\item \textbf{Enfield} liefert das \textit{Meta-Model}, welches für die Beschreibung der gewünschten Appliktion benötigt wird.
	\item \textbf{Norbury} stellt für \textit{Server}-seitige Applikationen den Plattform Code bereit.
	\item \textbf{Dalston} ist ein Software-Generator, welcher den \textit{Server}-seitigen Code in Java generiert.
	\item \textbf{Welling} ist ein Software-Generator, welcher Android Applikationen generiert.
	\item \textbf{Purley} ist ein Software-Generator, welcher Polymer Webkomponenten generiert. 
\end{itemize}

Diese Arbeit behandelt das Design und die Entwicklung von \textbf{Welling}.
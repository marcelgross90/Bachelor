\chapter{Grundlagen}\label{ch:basics}
\section{\acf{rest}}\label{sec:rest}
In dem generierten Projekt, sollen alle benötigten Inforamtionen mittels \ac{rest} von dem zugehörigen, generierten Backend geladen werden. 

\ac{rest} \cite{rest_fielding} ist ein Programmierparadigma, welches sich auf folgende Prinzipien stützt:


\begin{enumerate}
	\item  Client-Server
	\item  Zustandslose Kommunikation
	\item  Caching
	\item  Uniform Interface
	\item  Layered System
	\item  Code-on-Demand (optional)
\end{enumerate}

\subsection{\acf{hateoas}}\label{sec:hateoas}
\ac{hateoas} fällt unter das Prinzip Uniform Interface. Es beschreibt, wie mit Hilfe eines endlichen Automaten eine \ac{rest}-Architektur entworfen werden kann.

Der Architekt einer \ac{rest}-konformen \acf{api} überlegt sich im vorraus, wie der Applikation-Fluss in der späteren Anwendung aussehen soll. Dafür definiert er verschiedene States und welche Transitionen zum nächsten State führen.

Als ein State kann beispielsweise das Anzeigen aller Lecturer in einer Campus-Applikation angesehen werden.
Die Transition hingegen ist zum Beispiel ein Link im Link-Header der Antwort, oder ein Attribut, der empfangen Ressource. 

Wird das \ac{api} mit Hilfe eines endlichen Automaten entwickelt, kann diese dem Client-Entwickler als Anleitung zum erstellen seines Clients dienen. Er benötigt lediglich einen \acf{url}, welche auf den initialen State des endlichen Automaten führt. Dieser liefert dann alle, zu diesem Zeitpunkt möglichen, Transitionen zurück. Mit Hilfe dieser Transaktionen, kann sich der Entwickler dann zum nächsten State bewegen. Auch dieser State liefert alle möglichen weiteren Transaktionen zurück. 
Wenn der Entwickler sich so durch die States bewegt, bekommt er die benötigten Informationen zum Aufbau und Ablauf der Applikation.

Die Abbildung \ref{finite_state_machine} zeigt einen solchen Automaten. Der Einstiegspunkt ist der State \enquote{Dispatcher} dieser liefert die Tansition zum State \enquote{Collection} zurück. Dieser State, verfügt über alle Informationen die benötigt werden um eine Collection der betroffenen Ressource anzuzeigen, weiterhin verfügt er auch das Wissen, über die beiden nächsten Transitionen zu den States \enquote{Create} und \enquote{Single}. Wie der Name des States annehmen lässt, wird der State \enquote{Create} benötigt um eine neue Ressource anzulegen. Von diesem State aus kann die Anwendung nur zurück zum State \enquote{Collection}. Der State \enquote{Single} enthält alle benötigten Daten um eine einzelne Ressource anzuzeigen. Vom hier kann die Anwendung zum State \enquote{Update} oder \enquote{Delete} wechseln. Der State \enquote{Update} ermöglicht es die Ressource zu bearbeiten. Von hier kann der Nutzer der Anwendung nur zum State \enquote{Single} zurückkehren. Der State \enquote{Delete} löscht die aktuelle Ressource und liefert die Transition zum State \enquote{Collection} zurück. 
Dieses Beispiel verdeutlich nocheinmal bildlich, das der Entwickler nur den Einstiegspunkt \enquote{Dispatcher} kennen muss. Die Anwendung liefert selbst alle benötigten Informationen um die Daten für die Anwendung nachzuladen.

\begin{figure}[H]
	\begin{center}
		\begin{tikzpicture}[->,>=stealth',shorten >=1pt,auto,node distance=3cm,
		semithick]
		\tikzstyle{every state}=[fill=none,text=black]
		
		\node[initial,state] (A) [minimum width=3cm]             {Dispatcher};
		\node[state]         (B) [right of=A, minimum width=2cm] {Collection};
		\node[state]         (D) [below of=B, minimum width=2cm] {Create};
		\node[state]         (C) [right of=B, minimum width=2cm] {Single};
		\node[state]         (E) [above of=C, minimum width=2cm] {Delete};
		\node[state]		 (F) [below of=C, minimum width=2cm] {Update};
		
		\path 
		(A) edge             (B)
		(B) edge [bend left] (C)
			edge [bend left] (D)
		(C)	edge 			 (E)
			edge [bend left] (B)
			edge [bend left] (F)
		(D) edge [bend left] (B)
		(E) edge   			 (B)
		(F) edge [bend left] (C);
		
		\end{tikzpicture}
			\caption[Aufbau eines \acl{rest}-\acl{api} mit Hilfe eines endlichen Automaten.]{Aufbau eines \ac{rest}-\ac{api} mit Hilfe eines endlichen Automaten.}
		\label{fig:finite_state_machine}
	\end{center}
\end{figure}

\section{Android}\label{sec:android}
\subsection{Custom Views}
Noch mehr Text


\section{Generatoren}\label{sec:generators}
\subsection{\acf{gemara}}\label{sec:gemara}
\documentclass[12pt,oneside,a4paper,parskip]{scrbook}
\usepackage[utf8]{inputenc}
\usepackage[T1]{fontenc}
\usepackage{lmodern}
\usepackage{csquotes}
\usepackage[ngerman]{babel}
\usepackage{floatflt} 
\usepackage{subfigure}
\usepackage[pdftex]{graphicx}
\usepackage[hidelinks]{hyperref}
\usepackage{color}
\usepackage{amssymb}
\usepackage{textcomp}
\usepackage{nicefrac}
\usepackage{pdfpages}
\usepackage{float} 
\usepackage{pdflscape}
\usepackage{subfigure}
\usepackage{pdfpages}  
\usepackage[verbose]{placeins} 
\usepackage[nouppercase,headsepline,plainfootsepline]{scrpage2}
\usepackage{listings}		
\usepackage{xcolor}			
\usepackage{color}			
\usepackage{caption}		
\usepackage{subfigure}			
\usepackage{epstopdf}		
\usepackage{longtable}  
\usepackage{setspace}
\usepackage{booktabs}
\usepackage[style=numeric, backend=biber]{biblatex}
\bibliography{literatur2}
\usepackage[nolist]{acronym}
\usepackage{tikz}
\usetikzlibrary{arrows,automata}

%%%%%%%%%%%%%%%%%%%
%% definitions
%%%%%%%%%%%%%%%%%%%
\def\BaAuthor{Marcel Groß}
\def\BaTitle{Design und Implementierung eines Generators für Android View Komponenten}
\def\BaSupervisorOne{Prof.\ Dr.\ Peter Braun}
\def\BaSupervisorTwo{Prof.\ Dr.\ Steffen Heinzl}
\def\BaDeadline{31.03.2017}

\hypersetup{
pdfauthor={\BaAuthor},
pdftitle={\BaTitle},
pdfsubject={Subject},
pdfkeywords={Keywords}
}

%%%%%%%%%%%%%%%%%%%
%% configs to include
%%%%%%%%%%%%%%%%%%%
\colorlet{punct}{red!60!black}
\definecolor{background}{HTML}{EEEEEE}
\definecolor{delim}{RGB}{20,105,176}
\colorlet{numb}{magenta!60!black}

\definecolor{gray}{rgb}{0.4,0.4,0.4}
\definecolor{darkblue}{rgb}{0.0,0.0,0.6}
\definecolor{cyan}{rgb}{0.0,0.6,0.6}

\definecolor{pblue}{rgb}{0.13,0.13,1}
\definecolor{pgreen}{rgb}{0,0.5,0}
\definecolor{pred}{rgb}{0.9,0,0}
\definecolor{pgrey}{rgb}{0.46,0.45,0.48}

\lstset{
  basicstyle=\ttfamily,
  columns=fullflexible,
  showstringspaces=false,
  commentstyle=\color{gray}\upshape
  linewidth=\textwidth
}

\lstdefinelanguage{json}{
    basicstyle=\normalfont\ttfamily,
    numbers=left,
    numberstyle=\scriptsize,
    stepnumber=1,
    numbersep=8pt,
    showstringspaces=false,
    breaklines=true,
    backgroundcolor=\color{background},
    literate=
     *{0}{{{\color{numb}0}}}{1}
      {1}{{{\color{numb}1}}}{1}
      {2}{{{\color{numb}2}}}{1}
      {3}{{{\color{numb}3}}}{1}
      {4}{{{\color{numb}4}}}{1}
      {5}{{{\color{numb}5}}}{1}
      {6}{{{\color{numb}6}}}{1}
      {7}{{{\color{numb}7}}}{1}
      {8}{{{\color{numb}8}}}{1}
      {9}{{{\color{numb}9}}}{1}
      {:}{{{\color{punct}{:}}}}{1}
      {,}{{{\color{punct}{,}}}}{1}
      {\{}{{{\color{delim}{\{}}}}{1}
      {\}}{{{\color{delim}{\}}}}}{1}
      {[}{{{\color{delim}{[}}}}{1}
      {]}{{{\color{delim}{]}}}}{1},
}

\lstset{language=xml,
  morestring=[b]",
  morestring=[s]{>}{<},
  morecomment=[s]{<?}{?>},
  stringstyle=\color{black},
  numbers=left,
  numberstyle=\scriptsize,
  stepnumber=1,
  numbersep=8pt,
  identifierstyle=\color{darkblue},
  keywordstyle=\color{cyan},
  backgroundcolor=\color{background},
  morekeywords={xmlns,version,type}% list your attributes here
}

\lstset{language=Java,
  showspaces=false,
  showtabs=false,
  tabsize=4,
  breaklines=true,
  keepspaces=true,      
  numbers=left,
  numberstyle=\scriptsize,
  stepnumber=1,
  numbersep=8pt,
  showstringspaces=false,
  breakatwhitespace=true,
  commentstyle=\color{pgreen},
  keywordstyle=\color{pblue},
  stringstyle=\color{pred},
  basicstyle=\ttfamily,
  backgroundcolor=\color{background},
%  moredelim=[il][\textcolor{pgrey}]{$$},
%  moredelim=[is][\textcolor{pgrey}]{\%\%}{\%\%}
}

\begin{document}

\begin{acronym}[Bash]
	\acro{mvc}[MVC]{Model View Controller}
	\acro{jar}[JAR]{Java Archive}
	\acrodefplural{jar}[JARs]{Java Archiven}
	\acro{yaml}[YAML]{YAML Ain’t Markup Language}
	\acro{json}[JSON]{JavaScript Object Notation}
	\acro{sql}[SQL]{Structured Query Language}
	\acro{html}[HTML]{HyperText Markup Language}
	\acro{dsl}[DSL]{Domänenspezifische Sprache}
	\acrodefplural{dsl}[DSLs]{Domänenspezifische Sprachen}
	\acro{url}[URL]{Uniform Resource Locator}
	\acro{api}[API]{Application Programming Interface}
	\acro{gemara}[GeMARA]{GEnerierung von Mobilen Applikationen basierend auf REST Architekturen}
	\acro{rest}[REST]{REpresentational State Transfer}
	\acro{ui}[UI]{User Interface}
	\acro{dsl}[DSL]{domänenspezifische Sprache}
	\acro{hateoas}[HATEOAS]{Hypermedia as the Engine of Application State}
\end{acronym}
%%%%%%%%%%%%%%%%%%%
%% Titelseite
%%%%%%%%%%%%%%%%%%%


\frontmatter
\titlehead{%  {\centering Seitenkopf}
  {Hochschule für angewandte Wissenschaften Würzburg-Schweinfurt\\
   Fakultät Informatik und Wirtschaftsinformatik}}
\subject{Bachelorarbeit}
\title{\BaTitle\\[15mm]}
\subtitle{\normalsize{vorgelegt an der Hochschule f\"{u}r angewandte Wissenschaften W\"{u}rzburg-Schweinfurt in der Fakult\"{a}t Informatik und Wirtschaftsinformatik zum Abschluss eines Studiums im Studiengang Informatik}}
\author{\BaAuthor}
\date{\normalsize{Eingereicht am: \BaDeadline}}
\publishers{
  \normalsize{Erstpr\"{u}fer: \BaSupervisorOne}\\
  \normalsize{Zweitpr\"{u}fer: \BaSupervisorTwo}\\
}

%\uppertitleback{ }
%\lowertitleback{ }

\maketitle


%%%%%%%%%%%%%%%%%%%
%% abstract
%%%%%%%%%%%%%%%%%%%

\section*{Zusammenfassung}
Der wachsende Markt für Android Applikationen erfordert immer schnellere Entwicklungs- und Releasezeiten. Viele dieser Anwendungen haben den gleichen Grundaufbau, sie ermöglichen Daten in Listen oder in einer Detailansicht anzuzeigen und gegebenenfalls diese Daten zu bearbeiten. Durch diesen ähnlichen Grundaufbau ist es möglich diese Applikationen durch Software-Generatoren erzeugen zu lassen.
Die Erzeugung von Android Applikationen mit Hilfe eines Software-Generators birgt einige Vorteile. So können durch Zuhilfenahme von Generatoren diese Applikationen mit sehr wenig Quellcode beschrieben und erzeugt werden. Der Schwerpunkt dieser Arbeit liegt darauf ein grundsätzliches Verständnis für die Entwicklung von CustomViews in Android und des Design eines Software-Generators.

\section*{Abstract}
The growing market for Android applications requires faster development and release times. 
Many of these applications have the same basic structure, they allow you to display data in lists or in a detailed view and, if necessary, to edit this data. Because of this similar basic structure, it is possible to have these applications generated by software generators.
The generation of Android applications using a software generator has some advantages. By using generators, these applications can be described and generated with very little source code. The main focus of this thesis is a basic understanding of the development of CustomViews in Android and the design of a software generator.

\newpage
\chapter*{Danksagung}

Ich möchte mich bei meinem betreuenden Professor Dr. Peter Braun bedanken. Er hatte immer ein offenes Ohr für mich und ich konnte mit ihm über auftretende Probleme diskutieren und diese dadurch lösen. Durch seine Erfahrung bei der Entwicklung von \acf{gemara} konnte er mir wertvolle Ratschläge und Ansätze für meine Implementierung geben.

Auch möchte ich Markus Fisch danken, der mir bei Problemen mit der Entwicklung der Android Applikation immer zur Seite stand. Dankenswerterweise hat er auch die ein oder andere Stunde mit mir die Applikation debugged, um noch die kleinsten Fehler zu finden. Egal ob es ein Fehler im generellen Aufbau der Anwendung war oder im Programmcode.

Zu Letzt möchte ich bei allen weiteren Unterstützern bedanken. Ob es Hilfe bei Problemen mit LaTeX war oder beim Rat zum Aufbau der Arbeit. Auch alle Korrekturleser möchte ich nochmal ein besonders danken. Ohne euch würde diese Arbeit nicht so flüssig sein.

%%%%%%%%%%%%%%%%%%%
%% Inhaltsverzeichnis
%%%%%%%%%%%%%%%%%%%
\tableofcontents										



%%%%%%%%%%%%%%%%%%%
%% Main part of the thesis
%%%%%%%%%%%%%%%%%%%
\mainmatter

\chapter{Einführung}\label{ch:intro}

Das Smartphone ist heutzutage der stete Begleiter eines Menschen. \enquote{Zwei Drittel der Bevölkerung und nahezu jeder 14- bis 29-Jährige geht darüber ins Netz.} \cite{usage} Auch die Prognose zeigt, das der Absatzmarkt immer weiter steigen wird (Abbildung \ref{fig:prognose_fig}).

\begin{figure}[H]
	\begin{center}
		\includegraphics[width=0.86\textwidth]{images/prognose-zur-anzahl-der-smartphone-nutzer-weltweit-bis-2020.png}
		\caption{Prognose zur Anzahl der Smartphone-Nutzer weltweit von 2012 bis 2020 (in Milliarden) \cite{prognose}.}
		\label{fig:prognose_fig}
	\end{center}
\end{figure}

Umso wichtiger ist es das die Softwareentwicklung diesen Trend ernst nimmt. Der ehemalige Google-Chef Eric Schmidt sagte bereits 2010: \enquote{Googles Devise heißt jetzt \enquote{Mobile first}}. 
Diese Devise wird von vielen Unternehmen verfolgt, das ist der Grund weswegen in den einzelnen Stores heutzutage so viele Apps angeboten werden. Bei Android im Playstore sind es im Oktober 2016 ca. 2,4 Millionen Apps \cite{play_store}, bei Apple im App Store sind es ca. 2 Millionen Apps (Stand Juni 2016) \cite{app_store}. Neben Googles Android und Apples iOs gibt es noch andere Betriebssysteme, beispielsweise Microsofts Windows Phone oder Blackberrys Blackberrys OS. Jedoch bestimmen die beiden erstgenannten Systeme den Markt (Abbildung \ref{fig:os_fig}).

\begin{figure}[H]
	\begin{center}
		\includegraphics[width=0.86\textwidth]{images/os.jpg}
		\caption{Der weltweite Marktanteil von Smartphone-Betriebssysteme \cite{os}.}
		\label{fig:os_fig}
	\end{center}
\end{figure}

Jede dieser Applikationen wurden einzeln für sich entwickelt und implementiert. Bei jedem Update zum Beispiel des Systems, müssen alle Anwendungen gewartet und überarbeitet werden, um die volle Funktionalität zu gewährleisten.

Würden einige Applikationen jedoch genauer analysiert werden, wäre das Ergebnis, dass in jeder dieser Anwendungen Codepassagen vorhanden sind, welche einen ähnlichen beziehungsweise den selben Zweck erfüllen. Werden diese Stellen im Programmcode abstrahiert, gibt es die Möglichkeit diese generieren zu lassen. Um Code generieren zu lassen, benötigt man so genannte Code-Generatoren. 

Im Bereich der Backend-Entwicklung gibt es bereits verschiedene Projekte die sich damit befassen. Ein Beispiel wäre der \textit{CRUD Admin Generator} \cite{generators}. Die Hochschule für angewandte Wissenschaften Würzburg-Schweinfurt entwickelt  unter der Leitung von Prof. Dr. Peter Braun auch einen Code-Generator unter dem Namen: \ac{gemara}. Mit Hilfe solcher Generatoren für den Bereich von Mobilen Applikationen, könnte der Entwicklungs- und Wartungsaufwand reduziert werden. 

Führt ein Systemupdate dazu, dass die Implementierung von verschiedenen Anforderungen nicht weiter funktionsfähig ist, muss dies nur einmalig an der entsprechenden Stelle im Code-Generator geändert werden und nicht in jeder Applikation einzeln. 

\section{Motivation}\label{sec:motivation}
Im Rahmen des Projektes \ac{gemara} gab es bereits Arbeiten, welche sich mit dem Thema der Generierung von \textit{Android Aktivities} beschäftigt. Die dabei entstandenen Lösungen, resultieren darin, das dass generieren von \textit{Aktivities} zu Problemen führt. Deshalb beschäftigt sich diese Ausarbeitung damit, nicht eine komplette \textit{Aktivity} zu erzeugen, sondern sogenannte Komponenten.

Eine Komponente, ist im wesentlichen eine kleine Anwendung für sich, welche nur eine einzige Aufgabe erfüllt. Dies könnte zum Beispiel das Anzeigen eines Dozenten in einer Campus-Applikation sein.
Aus den erzeugten Komponenten, kann eine Art Bausatz entstehen. Mit dessen Hilfe der Entwickler seine Applikation zusammen bauen kann. Dabei wird ihm freie Wahl gelassen, wie der Aufbau seiner Anwendung aussieht, er bedient sich nur an gegebener Stelle an den Komponenten. Dadurch reduziert sich der Entwicklungsaufwand für ihn.

Bewegen wir uns in der Domain einer Hochschule, so kann eine Bibliothek mit den erzeugten Komponenten allen Studierenden zur Verfügung gestellt werden. Dadurch wäre jeder dieser Studierenden in der Lage eine persönliche Campus-Applikation zu entwickeln. Durch die einzelnen Komponenten kann dann sichergestellt werden, dass grundsätzliche Funktionalität bereits gewährleistet ist.

\section{Zielsetzung}\label{sec:target}
Ziel dieser Ausarbeitung liegt darin, dass der Leser ein grundsätzliches Verständnis für die Entwicklung von Android Applikationen beziehungsweise Android-Bibliotheken vermittelt bekommt. Weiterhin soll das Wissen über Datenkommunikation mittels \textit{\ac{rest}} vertieft werden. Hierbei wird ein Schwerpunkt auf das \textit{Hypermedia-Prinzip} gelegt. 

Neben diesen spezifischen Anforderungen, soll ein Verständnis für der Implementierung von Generatoren entstehen. Dafür muss der Entwickler entscheiden können, was von der Implementierung als statischer Code angesehen werden kann und welcher generisch ist. Dieses Verständnis ist wichtig, um die Komplexität der Generatoren zu reduzieren. Da die statischen Anteile jedes mal identisch sind.
Auch soll auf die Frage eingegangen werden, ob man das \textit{\ac{ui}}, welches ebenfalls generiert wird, auch generisch gestalten kann. Das bedeutet, dass nicht nur die Informationen, welche angezeigt werden sollen beschreibt. Sondern auch wie diese angezeigt werden sollen.

Wenn es möglich ist dass das \textit{\ac{ui}} als Teil der \textit{\acf{dsl}} beschrieben werden kann, so hat der Nutzer des entsprechenden Generators die Freiheit, selbst zu entscheiden ob zum Beispiel bei seiner Campus-App, bei der Liste aller Dozenten das Profilbild links oder rechts angezeigt werden soll.

\section{Aufbau der Arbeit}\label{sec:structure}
Diese Ausarbeitung ist in sieben Kapitel unterteilt. In der Einführung wird zu Beginn auf den Stellenwert von Android Applikationen eingegangen. Darauf folgt die Motivation, weswegen diese Arbeit geschrieben wurde. Diese soll die Problemstellung anreißen und zur Zielsetzung hinführen. In diesem Teil der Einführung wird definiert, was der Sinn dieser Arbeit ist. Das Kapitel wird dann mit dem Bereich abgeschlossen, welchen den Aufbau der Arbeit beschreibt.

Das zweite Kapitel befasst sich mit den Grundlagen. Hier soll der Leser noch einmal seinen Kenntnisstand über \textit{\acf{rest}} auffrischen und die Bedeutung von \textit{\acf{hateoas}} verstehen. Neben dem Bereich der Netzwerkkommunikation wird außerdem noch der Bereich Android angeschnitten. Hier liegt der Schwerpunkt in der Entwicklung und \textit{CustomViews}. Dabei werden die einzelnen Schritte aufgezeigt, die benötigt werden um diese Komponenten zu erstellen und zu benutzen. Der letzte Teil in den Grundlagen befasst sich mit Software-Generatoren. Hier wird dem Leser kurz erklärt was eine \textit{\acf{dsl}} ist und welche zwei generelle Arten es gibt.
Zum Anschluss wird das Projekt \acf{gemara} vorgestellt. 

Im folgenden Kapitel wird die Problemstellung behandelt, dabei wird die Referenzanwendung vorgestellt. Diese Vorstellung inkludiert sowohl das Backend sowie die Android Applikation. Es wird darauf eingegangen das das Backend mit Hilfe von \ac{gemara} generiert wurde und wie das daraus resultierende \acl{api} aussieht. Anschließend wird die Anwendung dahingehend analysiert, dass eine Einteilung in generischen sowie spezifischen Quellcode vorgenommen werden kann. Auch werden die einzelnen \textit{Views} analysiert. Diese Analyse sollen Gemeinsamkeiten beim Aufbau und Design offenbaren. So das diese mit einer \textit{\ac{dsl}} beschrieben werden können. Zum Abschluss wird das \textit{Meta-Modell} vorgestellt. Es wird auf die Anforderungen an dieses eingegangen und anschließend zwei Modelle vorgestellt. Ein reines Android Modell und dann die Erweiterung des vorhandenen Enfield-Modells. Der nächste Bereich befasst sich mit der Analyse, hierbei soll herausgefunden werden welche Daten das \textit{Meta-Modell} überhaupt benötigt. In der Analyse werden die einzelnen \textit{Views} genauer betrachtet und es wird ein Augenmerk auf den Programmablauf und die Aktionen bei Klick gelegt. Nach der Analyse wird der Aufbau der \textit{View-Meta-Modelle} vorgestellt. Bei jeder \textit{View} wird auf deren Besonderheit und Möglichkeiten eingegangen. Neben den \textit{View-spezifischen} Daten wird auch noch aufgezeigt, welche Dateien allgemein benötigt werden und wo diese im vorgegebenen Modell platziert werden sollten. 

Das Kapitel Lösung stellt \textit{Welling} vor. Dies ist der in dieser Arbeit entwickelte Software-Generator. Zunächst wird das \textit{Java \acf{api} JavaPoet} im Zusammenhang mit der Generierung von Java-Klassen vorgestellt, anschießend wird beschrieben, wie andere Datei-Typen generiert werden können. Es wird gezeigt welche Features unterstützt werden müssen. Nachfolgend wird der Aufbau des Generators vorgestellt. Es wird auf die einzelnen Bereiche eingegangen und deren Aufgabe sowie Funktionsweise erklärt. Abgeschlossen wird das Kapitel mit einer Anleitung, wie die generierte Applikation gebaut und ausgeführt werden kann.

Das fünfte Kapitel, Evaluierung anhand einer Beispielanwendung, ist in drei Bereiche eingeteilt. Am Anfang wird die Beispielanwendung vorgestellt. Anschließend wird auf die Erstellung und Nutzung des \textit{Meta-Modells} eingegangen, wobei hier auch Einschränkungen durch dieses aufgezeigt werden. Der letzte Bereich befasst sich dann noch einmal mit den Zeitaufwänden und der Komplexität der Entwicklung, Wartung und Nutzung des Generators. Auch wird die Komplexität der erzeugen Applikation kritisch bewertet.

Im letzten Kapitel, Zusammenfassung, wird die Arbeit noch einmal Revue passieren lassen. Zusätzlich noch mögliche Erweiterungen und Ergänzungen an \textit{Meta-Modell} und Software-Generator vorgestellt.

\chapter{Grundlagen}\label{ch:basics}
\section{\acf{rest}}\label{sec:rest}
In dem generierten Projekt, sollen alle benötigten Daten mittels \ac{rest} von dem zugehörigen, generierten Backend geladen werden. 

\ac{rest} \cite{rest_fielding} ist ein Programmierparadigma, welches sich auf folgende Prinzipien stützt:


\begin{enumerate}
	\item  Client-Server
	\item  Zustandslose Kommunikation
	\item  Caching
	\item  Uniform Interface
	\item  Layered System
	\item  Code-on-Demand (optional)
\end{enumerate}

\subsection{\acf{hateoas}}\label{sec:hateoas}
\ac{hateoas} fällt unter das Prinzip Uniform Interface. Es beschreibt, wie mit Hilfe eines endlichen Automaten eine \ac{rest}-Architektur entworfen werden kann.

Der Architekt einer \ac{rest}-konformen \acf{api} überlegt sich im vorraus, wie der Applikation-Fluss in der späteren Anwendung aussehen soll. Dafür definiert er verschiedene States und welche Transitionen zum nächsten State führen.

Als ein State kann beispielsweise das Anzeigen aller Lecturer in einer Campus-Applikation angesehen werden.
Die Transition hingegen ist zum Beispiel ein Link im Link-Header der Antwort, oder ein Attribut, der empfangen Ressource. 

Wird das \ac{api} mit Hilfe eines endlichen Automaten entwickelt, kann diese dem Client-Entwickler als Anleitung zum erstellen seines Clients dienen. Er benötigt lediglich einen \acf{url}, welcher auf den initialen State des endlichen Automaten führt. Dieser liefert dann alle, zu diesem Zeitpunkt möglichen, Transitionen zurück. Mit Hilfe dieser Transitionen, kann sich der Entwickler dann zum nächsten State bewegen. Auch dieser State liefert neben den Ressourcen, alle möglichen weiteren Transaktionen zurück. 
Wenn der Entwickler sich so durch die States bewegt, bekommt er die benötigten Informationen zum Aufbau und Ablauf der Applikation.

Die Abbildung \ref{finite_state_machine} zeigt einen solchen Automaten. Der Einstiegspunkt ist der State \enquote{Dispatcher} dieser liefert die Tansition zum State \enquote{Collection} zurück. Dieser State, verfügt über alle Informationen die benötigt werden um eine Collection der betroffenen Ressource anzuzeigen, weiterhin verfügt er auch das Wissen, über die beiden nächsten Transitionen zu den States \enquote{Create} und \enquote{Single}. Wie der Name des States annehmen lässt, wird der State \enquote{Create} benötigt um eine neue Ressource anzulegen. Von diesem State aus kann die Anwendung nur zurück zum State \enquote{Collection}. Der State \enquote{Single} enthält alle benötigten Daten um eine einzelne Ressource anzuzeigen. Vom hier kann die Anwendung zum State \enquote{Update} oder \enquote{Delete} wechseln. Der State \enquote{Update} ermöglicht es die Ressource zu bearbeiten. Von hier kann der Nutzer der Anwendung nur zum State \enquote{Single} zurückkehren. Der State \enquote{Delete} löscht die aktuelle Ressource und liefert die Transition zum State \enquote{Collection} zurück. 
Dieses Beispiel verdeutlich nocheinmal bildlich, das der Entwickler nur den Einstiegspunkt \enquote{Dispatcher} kennen muss. Die Anwendung liefert selbst alle benötigten Informationen um die Daten für die Anwendung nachzuladen.

\begin{figure}[H]
	\begin{center}
		\begin{tikzpicture}[->,>=stealth',shorten >=1pt,auto,node distance=3cm,
		semithick]
		\tikzstyle{every state}=[fill=none,text=black]
		
		\node[initial,state] (A) [minimum width=3cm]             {Dispatcher};
		\node[state]         (B) [right of=A, minimum width=2cm] {Collection};
		\node[state]         (D) [below of=B, minimum width=2cm] {Create};
		\node[state]         (C) [right of=B, minimum width=2cm] {Single};
		\node[state]         (E) [above of=C, minimum width=2cm] {Delete};
		\node[state]		 (F) [below of=C, minimum width=2cm] {Update};
		
		\path 
		(A) edge             (B)
		(B) edge [bend left] (C)
			edge [bend left] (D)
		(C)	edge 			 (E)
			edge [bend left] (B)
			edge [bend left] (F)
		(D) edge [bend left] (B)
		(E) edge   			 (B)
		(F) edge [bend left] (C);
		
		\end{tikzpicture}
			\caption[Aufbau eines \acl{rest}-\acl{api} mit Hilfe eines endlichen Automaten.]{Aufbau eines \ac{rest}-\ac{api} mit Hilfe eines endlichen Automaten.}
		\label{fig:finite_state_machine}
	\end{center}
\end{figure}

\section{Android}\label{sec:android}
Die Software-Plattform Android basiert auf Linux und wird als Betriebsystem für mobile Endgeräte verwendet.
Das System wird als Open Source Projekt von der Open Handset Alliance entwickelt \cite{open_handset_alliance}. Dabei ist ein Ziel, die Schaffung eines offenen Standarts für mobile Endgeräte.
Die Entwicklung ist nicht abgeschlossen, die aktuelle Version ist 7.0 Nougat (Stand Feb. 2017).

Programme für diese Plattfrom nennt man Applikationen oder kurz Apps. Eine App stellt alle nötigen Sourcen bereit, zum Beispiel den Programmcode, Layout und Grafiken, die benötigt werden, um diese App auf einem Android-Endgerät auszuführen.

\subsection{Custom Views}
Mit Hilfe von Widgets und Layouts können Views definiert werden. Diese Views stellen dann die gewünschte Information auf dem Display dar. Die bekanntesten Widgets sind: TextView, Button und EditText. Die Anordnung dieser Widgets erfolgt dann mit einem Layout. Es gibt hierbei verschiendene Layouts zur Auswahl. Beispielsweise das LinearLayout, mit horizontaler oder vertikaler Orientierung. Ein weiteres Beispiel ist das RelativeLayout.

Reichen die Standart-Layouts und -Widgets nicht aus, gibt es noch die Möglichkeit eigene zu entwickeln. Dies ermöglicht diese Views um Attribute und Methoden zu erweitern. Diese können dann sowohl in der Layout-XML als auch im Programm-Code angesprochen werden.

Ausgehend davon das eine Applikation eine Liste von Personen mit Hilfe einer RecyclerView anzeigen soll, gibt es die Möglichkeit eine CardView zu erzeugen, welche eine einzelne Person darstellt. Diese CardView kann in einem XML-Layout wie allgemein bekannt definiert werden. Um die View dann mit den entsprechenden Informationen zu befüllen werden im Adapter der ListView dann die einzelnen Attribute einzeln angesprochen und mit den erforderlichen Details befüllt.

Alternativ besteht die Möglichkeit eine Custom-View zu erzeugen, in diesem Fall eine PersonCardView.
Hierfür sind folgende Schritte notwendig:

\subsubsection{Registrieren der Custom-View}
Zur Erzeugung und Registrierung von Custom-Views wird eine Datei \enquote{attrs.xml} benötigt. Diese wird liegt im Ordner \enquote{values} im Verzeichnis \enquote{res}. 
In dieser XML-Datei werden im \enquote{resoruces}-Bereich die einzelnen Custom-Views aufgelistet. Es besteht die Möglichkeit diesen Views zusätzlich Attribute zuzuweisen. Ein Attribut besteht dabei immer aus einem Namen und einem Format. Dieses Format definiert den erwarteten Eingabewert. Es gibt folgende definierte Formate: string, integer, boolean oder color. 
Formate können kombiniert werden. Beispielsweise das Attribut \enquote{backgroundColor} könnte so definiert werden format=\enquote{color|string}. Listing \ref{lst:attrs} zeigt den Aufbau einer \enquote{attrs.xml}-Datei.

\begin{lstlisting}[label=lst:attrs,
language=xml,
firstnumber=1,
caption=Aufbau einer \enquote{attrs.xml} - Datei]				   
<resources>
	<declare-styleable name="AttributeInput">
		<attr name="hintText" format="integer"/>
		<attr name="inputType" format="string"/>
	</declare-styleable>
	
	<declare-styleable name="PersonCardView" />
</resources>
\end{lstlisting}

\subsubsection{Definieren des Aufbaus der PersonCardView}

Da die PersonCardView eine Custom-View ist, welche aus verschiedenen Widgets zusammengesetzt wurde, müssen diese auch definiert werden. Dies geschieht wie gewohnt mit Hilfe einer XML-Datei, mit einer Ausnahme. Die Root-View ist in diesem Fall keine CardView sondern ein beliebiges anderes Layout. Da die PersonCardView von CardView erbt und somit bereits eine CardView ist.

\begin{lstlisting}[label=lst:personCardViewXml,
language=xml,
firstnumber=1,
caption=Aufbau der PersonCardView mit Hilfe einer XML-Datei]				   
<RelativeLayout xmlns:android="http://schemas.android.com/apk/res/android"
	android:id="@+id/relativeLayout"
	android:layout_width="match_parent"
	android:layout_height="wrap_content">

	<TextView
		android:id="@+id/first_name"
		android:layout_width="wrap_content"
		android:layout_height="wrap_content"
		android:text="@string/firstName"/>

	<TextView
		android:id="@+id/last_name"
		android:layout_width="wrap_content"
		android:layout_height="wrap_content"
		android:text="@string/last_name"/>

...

</RelativeLayout>
\end{lstlisting}

\subsubsection{Erzeugen einer PersonCardView Klasse}

Hierfür wird eine neue Java-Klasse erzeugt, welche von CardView erbt. Es kann auch direkt von der View-Klasse geerbt werden und anschließend mithilfe der Methode \enquote{onDraw}, welche überschrieben werden muss, den gewünschten Inhalt anzueigen. Sei es nun Text, Formen oder Benutzereingaben.
In diesem Fall entspricht die CardView weitestgehend bereits den Anforderungen, so das diese genutzt wird.
Die Vererbungsstruktur bringt mit sich, das die Konstruktoren der CardView implementiert werden müssen.
Die Anzahl dieser Konstruktoren hängt von der Minimum SDK-Versions des Projekts ab. Dieses Projekt nutzt das Minimum Level 12 somit müssen drei Konstrukotoren überschrieben werden. Ab einen Level von 21, sind es vier, da ein weiteres Attribut zur View hinzugefügt wurde.

\begin{lstlisting}[label=lst:personCardView,
language=java,
firstnumber=1,
caption=Konstruktoren der PersonCardView]				   
public class PersonCardView extends CardView {

public PersonCardView(Context context) {
	super(context);
	init(context, null, 0);
}

public PersonCardView(Context context, AttributeSet attrs) {
	super(context, attrs);
	init(context, attrs, 0);
}

public PersonCardView(Context context, AttributeSet attrs, int defStyleAttr) {
	super(context, attrs, defStyleAttr);
	init(context, attrs, defStyleAttr);
}
...
}
\end{lstlisting}

Innerhalb der \enquote{init}-Methode wird definiert, was die View anzeigen beziehungsweise was sie tun soll. 
In diesem Beispiel werden die verwendeten Widgets initialisiert. Besässe die PersonCardView noch eigene Attribute, so würden diese im AttributeSet übergeben und könnten daraus in ein TypedArray geschrieben werden. Dieses TypedArray muss am Ende \enquote{recycled} werden, damit es für einen späteren Aufruf wieder zur Verfügung steht.

Jetzt wird die PersonCardView um eine Methode \enquote{setPerson} erweitert. Diese ist angelehnt an die Methode \enquote{setText} der TextView. Sie ermöglicht das der PersonCardView ein Objekt Person übergeben wird und füllt dann die entsprechenden Widgets mit den dazugehörigen übergebenen Daten.

\begin{lstlisting}[label=lst:setPerson,
language=java,
firstnumber=1,
caption=\enquote{setPerson} - Methode aus der PersonCardView]				   
public void setPerson(Person person) {
	this.firstName.setText(person.getFirstName());
	this.lastName.setText(person.getLastName());	
	...
}
\end{lstlisting}

Mit Hilfe dieser Methode wird die Nutztung der PersonCardView vereinfacht. Im Adapter der RecyclerView wird jetzt nicht mehr jedes einezelne Widget definiert und mit Informationen befüllt. Sondern nur noch die PersonCardView und mit der \enquote{setPerson} - Methode kann die komplette Karte mit den Daten einer Peron mit nur einem Methodenaufruf befüllt werden.

\section{Generatoren}\label{sec:generators}
\subsection{\acf{gemara}}\label{sec:gemara}

\chapter{Problemstellung} \label{ch:problem}
In diesem Kapitel wird die Referenzanwendung für diese Arbeit vorgestellt. Zuerst wird das benötigte Backend vorgestellt. Es wird darauf eingegangen, das das benötigte \textit{\acf{api}} mit Hilfe von \acf{gemara} generiert wurde und wie das zugehörige \textit{Enfield-Meta-Modell} aussieht. Nach der Vorstellung des Modells wird auch das erzeugte \textit{\acf{api}} dargestellt. Anschließend wird die Android Applikation vorgestellt, wobei in diesem Bereich primär auf das \textit{\acf{ui}} eingegangen wird. Abschließend wird das \textit{Meta-Modell} und mögliche Erweiterungen vorgestellt und gegeneinander abgewogen.

Um einen Generator zu entwickeln, ist es hilfreich, eine solche Implementierung der gewünschten Applikation mit all ihren Funktionen und Anforderungen zu entwickeln. Bei einer anschließenden Quellcode-Analyse sollte darauf geachtet werden, dass die einzelnen Klassen weitestgehend abstrahiert sind und eine Einteilung in generischen und spezifischen Quellcode erfolgen kann. Der generische Quellcode ist einfacher zu generieren, da dieser statisch ist und sich für alle folgenden Implementierungen nicht verändert. 

\section{Vorstellung der Referenzanwendung}
Die Beispielanwendung soll dem Nutzer die Möglichkeit geben, die Dozenten der Fakultät Informatik der FHWS, und deren Ämter einzusehen, einen neuen Dozenten anzulegen, einen existierenden Dozenten zu bearbeiten oder zu löschen. Neben den Dozenten, soll es weiterhin möglich sein die Ämter eines Dozenten einzusehen, zu bearbeiten, neu anzulegen oder zu löschen. Für jede dieser Aktionen werden entsprechende Endpunkte in dem \textit{\acf{api}} benötigt. Jeder dieser Endpunkte benötigt einen Zugriff auf die darunter liegende Datenbank. Um das zu realisieren wird ein \textit{Backend-Projekt} mit Hilfe von \ac{gemara} erzeugt.

\subsection{Backend Referenzimplementierung}
Dieses Kapitel stellt die Referenzimplementierung des Backends für die Referenzanwendung vor. Dabei wird das \textit{Enfield-Meta-Modell} und das daraus resultierende \textit{\acf{api}} vorgestellt.

\subsubsection{\textit{Enfield-Meta-Modell} der Referenzimplementierung} \label{sec:enfield_intro}
Dieses Kapitel befasst sich mit dem benötigten \textit{Enfield-Meta-Modell}. Dafür werden Quellcode-Beispiele aufgezeigt und auf deren Besonderheiten eingegangen. In diesem Kapitel wird nicht das ganze Modell vorgestellt, nur die wichtigsten Aspekte daraus. Das komplette Modell kann im Anhang unter Listing \ref{lst:enfield_model} eingesehen werden.

In Listing \ref{lst:constructor_enfield} wird die Initialisierung des \textit{Enfield-Modells} dargestellt. Zusätzlich werden die Attribute \textit{producerName}, \textit{packagePrefix} und der Name des Projektes festgelegt.

\begin{lstlisting}[label=lst:constructor_enfield,
language=java,
firstnumber=1,
caption=Initialisierung des \textit{Enfield-Meta-Modells}.]	
public MyEnfieldModel() {
	this.metaModel = new Model();

	this.metaModel.setProducerName("fhws");
	this.metaModel.setPackagePrefix("de.fhws.applab.gemara");
	this.metaModel.setProjectName("Lecturer");
}
\end{lstlisting}

Das Listing \ref{lst:lecturer_res} zeigt wie eine Ressource angelegt wird, welche einen Dozenten darstellen sollen. Der Ressource wird einen Namen zugewiesen und bekommt einen \textit{MediaType}. Außerdem werden alle Attribute in dem sie benannt und einen Datentyp zugewiesen bekommen definiert.
 Da eines der Attribute eines Dozenten seine Ämter sind, welche als Subressource der \textit{SingleResource Lecturer} dargestellt ist, ist es wichtig, dass das \textit{Enfield-Modell} ebenfalls eine \textit{SingleResource} für diese Ämter besitzt. Dies wird hier durch den Methodenaufruf \textit{createSingleResourceCharge()} sichergestellt.

\newpage

\begin{lstlisting}[label=lst:lecturer_res,
language=java,
firstnumber=1,
caption=Erzeugung der \textit{SingleResource} \textit{Lecturer}. ]
this.metaModel.addSingleResource("Lecturer");

this.lecturerResource = this.metaModel.getSingleResource("Lecturer");

this.lecturerResource.setModel(this.metaModel);
this.lecturerResource.setResourceName("Lecturer");
this.lecturerResource.setMediaType(
	"application/vnd.fhws-lecturer.default+json");

final SimpleAttribute title = new SimpleAttribute("title", SimpleDatatype.STRING);
...
final SimpleAttribute roomNumber = new SimpleAttribute("roomNumber", SimpleDatatype.STRING);
final SimpleAttribute homepage = new SimpleAttribute("homepage", SimpleDatatype.LINK);

createSingleResourceCharge();
final ResourceCollectionAttribute charge = new ResourceCollectionAttribute("chargeUrl", this.chargeResource);

this.lecturerResource.addAttribute(title);
...
this.lecturerResource.addAttribute(charge);

addImageAttributeForLecturerResource();
\end{lstlisting}

Sind alle benötigten Ressourcen im Modell definiert, wird der \textit{endliche Automat} beschrieben angefangen mit dem \textit{DispatcherState} (\ref{lst:dispatcher_impl}). Dieser bekommt einen Namen, und wird als \textit{DispatcherState} dem Modell hinzugefügt.

\begin{lstlisting}[label=lst:dispatcher_impl,
language=java,
firstnumber=1,
caption=Erzeugung des \textit{DispatcherStates}. ]
final GetDispatcherState dispatcherState = new GetDispatcherState();
dispatcherState.setName("Dispatcher");
dispatcherState.setModel(this.metaModel);
this.metaModel.setDispatcherState(dispatcherState);
this.dispatcherState = dispatcherState;
\end{lstlisting}

\newpage

Beispielhaft für alle nachfolgenden \textit{States} zeigt Listing \ref{lst:getState_impl} wie der \textit{State GetAllLecturers} erzeugt wird.
Auch dieser \textit{State} bekommt einen Namen, daneben wird ihm die Ressource zugewiesen, welche er bedienen soll. Neben diesen Eigenschaften, werden dem \textit{State} alle \textit{Transitionen} hinzugefügt. In diesem Fall wird zusätzlich dem \textit{DispatcherState} die Information übergeben, das der \textit{GetAllLecturers State} sein \textit{Folgestate} ist. Alle weiteren \textit{States} werden analog definiert (siehe Listing \ref{lst:enfield_model}).

\begin{lstlisting}[label=lst:getState_impl,
language=java,
firstnumber=1,
caption=Erzeugung des \textit{GetAllLecturers States}. ]
final GetPrimaryCollectionResourceByQueryState getAllLecturersCollectionState = new GetPrimaryCollectionResourceByQueryState();
getAllLecturersCollectionState.setName("GetAllLecturers");
getAllLecturersCollectionState.setModel(this.metaModel);
getAllLecturersCollectionState.setResourceType(this.lecturerResource);

this.dispatcherState.addTransition(new ActionTransition(getAllLecturersCollectionState, "getAllLecturers"));
getAllLecturersCollectionState.addTransition(this.getLecturerByIdState);

this.metaModel.addState(getAllLecturersCollectionState.getName(), getAllLecturersCollectionState);

this.getCollectionOfLecturersState = getAllLecturersCollectionState;
\end{lstlisting}
 
\subsubsection{Vorstellung des \textit{\acf{api}}}
Mit Hilfe des zuvor beschriebenen \textit{Enfield-Meta-Modell} wird ein \textit{\acl{api}} generiert, welche in diesem Kapitel vorgestellt wird.
Die Abbildung \ref{fig:api} zeigt das \textit{\acl{api}}, welches für die Beispielanwendung benötigt wird. Dieses\textit{ \ac{api}} entspricht einem \textit{endlichen Automaten} und spiegelt alle Funktionen der Applikation wieder.  

\begin{figure}[H]
	\begin{center}
		\includegraphics[width=\textwidth]{images/api.png}
		\caption{Darstellung des \textit{\ac{api}} der Beispielanwendung.}
		\label{fig:api}
	\end{center}
\end{figure}

\subsection{Android Referenzimplementierung}\label{sec:ref_impl}
Nach der Vorstellung des Backends, geht dieses Kapitel darauf ein wie die Informationen des \textit{\acl{api}s} in die Android Applikation einfließt.
Für die Realisierung der gewünschten Funktion werden folgende \textit{Views} benögt: eine \textit{RecyclerView}, welche alle Dozenten in jeweils einer eigenen \textit{CardView} darstellen, eine \textit{DetailView} diese zeigt einen einzelnen Dozenten und all seine Informationen, jeweils eine \textit{InputView} zur Erzeugung eines neuen Dozenten beziehungsweise zum bearbeiten eines existierenden Dozenten, eine \textit{RecyclerView}, welche die Ämter eines Dozenten anzeigt, die Detail-Ansicht eines Amtes sowie wiederum jeweils eine \textit{View} zum bearbeiten beziehungsweise zur Neuanlage eines Amtes.

\subsubsection{\textit{\acf{ui}} der Referenzimplementierung}
In diesem Kapitel wird das \textit{\acl{ui}} der Android Applikation vorgestellt. Dabei wird gezeigt wie die einzelnen \textit{Views} umgesetzt werden.
Wie dem \acl{api} entnommen werden kann, steigt der Nutzer mit der Liste aller Dozenten in die Applikation ein. Diese Liste ist in diesem Fall wie in der Einleitung beschrieben mit einer \textit{RecyclerView} und einzlenen \textit{CardViews} realisiert. Diese Liste kann in Abbildung \ref{fig:list} eingesehen werden.

\begin{figure}[H]
	\begin{center}
		\includegraphics[width=0.4\textwidth]{images/list.png}
		\caption{\textit{RecyclerView} zur Darstellung aller Dozenten.}
		\label{fig:list}
	\end{center}
\end{figure}

Über den \textit{Plus-Button} auf der linken oberen Seite in der \textit{View} kommt der Nutzer zu der \textit{View}, welche es ermöglicht, einen neuen Dozenten anzulegen. Diese \textit{View} besteht aus \textit{EditText}-Feldern, welche Vorgeben welche Informationen zur Neuanlage benötigt werden. Die Abbildung \ref{fig:input_view} zeigt diese \textit{View}. Diese \textit{View} validiert auch ob eine Eingabe getätigt wurde, andernfalls wird eine Fehlermeldung angezeigt. Diese Darstellung der Fehlermeldung ist in Abbildung \ref{fig:input_error} dargestellt.

\begin{figure}[H]
	\begin{center}
		\includegraphics[width=0.4\textwidth]{images/input_small.png}
		\caption{Ausschnitt der \textit{View} zur Erstellung eines Dozenten.}
		\label{fig:input_view}
	\end{center}
\end{figure}

\begin{figure}[H]
	\begin{center}
		\includegraphics[width=0.4\textwidth]{images/input_error.png}
		\caption{Fehlermeldung bei der Neuanlage eines Dozenten.}
		\label{fig:input_error}
	\end{center}
\end{figure}

Durch die Neuanlage eine Dozenten oder durch den Klick auf deine Karte in der Liste, wird der Nutzer auf die Detailansicht eines Dozenten weitergeleitet (Abbildung \ref{fig:detail_view}). Hier bekommt der Nutzer die Möglichkeit detaillierte Informationen zum betroffenen Dozenten zu bekommen. Des weiteren bekommt er die Möglichkeit den aktuellen Dozenten zu bearbeiten oder diesen zu löschen. Diese Aktionen können über das Kontextmenü aufgerufen werden. Wobei die \textit{View} zum editieren des Dozenten anlog der \textit{View} zur Neuanlage mit der Ausnahme, das die vorhandenen Daten bereits vor ausgefüllt sind, aussieht. Wird das Löschen des Dozenten über einen \textit{Dialog} realisiert. Dieser \textit{Dialog} ist in Abbildung \ref{fig:dialog} abgebildet.

\begin{figure}[H]
	\begin{center}
		\includegraphics[width=0.4\textwidth]{images/detail.png}
		\caption{Detailansicht eines Dozenten.}
		\label{fig:detail_view}
	\end{center}
\end{figure}

\begin{figure}[H]
	\begin{center}
		\includegraphics[width=0.4\textwidth]{images/dialog.png}
		\caption{\textit{Dialog} zum Löschen eines Dozenten.}
		\label{fig:dialog}
	\end{center}
\end{figure}

Über den \textit{Charge-Button} gelangt der Nutzer zur Liste mit den Ämtern des Dozenten. Die \textit{Views} für diese Ämter sind analog zu denen der Dozenten. Mit der Ausnahme, dass bei der Neuanlage, beziehungsweise bei dem Bearbeiten eines Amtes dieses mal nicht ausschließlich \textit{EditText} zur Verfügung steht. Da die Ämter die zwei Datumsattribute für den Start und das Ende besitzen, so wurde hierfür das \textit{DateTimePicker-Widget} eingebaut. Dieses kann in Abbildung \ref{fig:date} eingesehen werden.

\begin{figure}[H]
	\begin{center}
		\includegraphics[width=0.4\textwidth]{images/date.png}
		\caption{\textit{DateTimePicker-Widget} zur Datumsauswahl.}
		\label{fig:date}
	\end{center}
\end{figure}

\subsubsection{Zahlen und Fakten}
In diesem Kapitel soll die Referenzimplementierung statistisch greifbar vorgestellt werden. So lässt sich die Applikation in zwei Bereiche einteilen, in die Applikation an sich, dies besitzt den kompletten spezifischen Code und eine Bibliothek. Diese beinhaltet den kompletten generischen Code sowie die \textit{CustomViews}. Die gesamte Anwendung besitzt ungefähr 3000 \textit{Lines of Java Code} und circa 1000 Zeilen an \textit{XML} Code. 

Die Applikation an sich ist der kleinere Teil der Implementierung, sie enthält elf Java Klassen, wobei es sich dabei um vier \textit{Activities} und sieben \textit{Fragments} handelt. Daneben besitzt sie sechs Layout \textit{XML}-Dateien und zwei Animations \textit{XML}-Dateien. Daneben noch die üblichen \textit{XML}- und \textit{Gradle}-Dateien.

Die Bibliothek an ist mit 44 Java Klassen wesentlich größer, wobei hiervon 15 Klassen spezifischen Code enthalten. Die Klassen lassen sich in \textit{abstrakte Activities}, \textit{abstrakte Fragments}, \textit{abstrakte Adapter}, \textit{abstrakte CustomViews}, \textit{abstrakte Models}, \textit{abstrakte Viewholder},  Klassen für die Netzwerkkommunikation, \textit{Adapter}, \textit{CustomViews}, \textit{Models} und \textit{Viewholder} einteilen. Neben den Java Klassen besitzt die Bibliothek 14 Layout- und drei Menü-Klassen. Diese 17 Klassen sind alles \textit{XML}-Dateien. Auch die Bibliothek besitzt die weiteren üblichen \textit{XML}- und \textit{Gradle}-Dateien.

\section{Analyse der Android Anwendung}
Dieses Kapitel beschäftigt sich mit der Android Referenzimplementierung. Es wird der Aufbau der Applikation vorgestellt und anhand dessen analysiert, welcher Programmcode als \textit{Plattformcode} und welche Programmcode generiert werden muss. Es werden auch ein paar Kennzahlen vorgestellt, um ein besseres Bild der Komplexität der Applikation zu erhalten. Anschließend werden die einzelnen \textit{Views} analysiert. Diese Analyse beschäftigt sich mit dem Design und den Funktionen der \textit{Views}.

\subsection{Analyse des Aufbaus}
Wird der Aufbau der Referenzimplementierung analysiert, so fällt auf das es möglich ist die meisten Klassen soweit zu abstrahieren, das diese keine Projekt spezifischen Informationen mehr enthalten. Im Falle dieser Referenzimplementierung keine Informationen zu Dozenten oder deren Ämter. Diese Klassen ohne diese spezifischen Informationen, wird im laufenden als \textit{Plattformcode} oder generischer Code bezeichnet. Das Ziel bei der Referenzimplementierung ist es, möglichst viel \textit{Plattformcode} und möglichst wenig spezifischen Code in der Anwendung zu haben. 

Das Schaubild \ref{fig:lecturer_structure} verdeutlicht das Verhältnis von generischen (weiße Kästen) und spezifischen (rote Kästen) Klassen. Die Anzahl der gleichbleibenden Klassen ist mit etwa 60 Prozent bereits höher als der Anteil an spezifischen Klassen. Je höher der Anteil dieser unveränderlichen Klassen, desto geringer wird die Komplexität des Generators. Da der Aufwand eine spezifische Klasse zu erzeugen mehr Logik benötigt, als eine Klasse, welche immer gleich bleibt.

\begin{figure}[H]
	\begin{center}
		\includegraphics[width=\textwidth, angle=90]{images/Lecturer.png}
		\caption{Aufbau der Referenzimplementierung.}
		\label{fig:lecturer_structure}
	\end{center}
\end{figure}

Daneben zeigt die Abbildung, auch noch die Aufteilung der Klassen in Klassen der Applikation (gestrichelte Kästen) und Klassen der Bibliothek (solide Kästen). Die Applikation an sich besteht nur aus ein paar wenigen \textit{Fragmenten} und \textit{Aktivities}, welche alle projektspezifisch sind. Der komplette generische Quellcode befindet sich in der Bibliothek. Des weiteren befinden sich dort auch die spezifischen Komponenten, beispielsweise der \textit{LecturerInputView}. Diese Komponente, kann in den Fragmenten zur Bearbeitung oder Neuanlage eines Dozenten dann mit wenigen Zeilen Programmcode verwendet werden.

Diese Art der Aufteilung ermöglicht es das ein Applikation Entwickler sich die Komponente, für das Anzeigen, Bearbeiten, Löschen und der Neuanlage generieren lassen kann. Diese Komponenten jedoch beliebig in seiner eigenen Applikation verwenden kann.


\subsection{Analyse der Android \textit{Views}}
In diesem Kapitel sollen die \textit{Views} der Android Applikation an sich analysiert werden. Hierfür werden die Bereiche: Aufbau der \textit{Views}, Darstellung von Schrift, und Aktionen bei Klick genauer betrachtet. Um ein \textit{Meta-Modell} für Android Anwendungen zu entwickeln, muss der Designer untersuchen, welche Eigenschaften dieses Modell besitzen soll. Diese Eigenschaften spiegeln die Möglichkeiten wieder, die Android Anwendung zu beschreiben. Für das Extrahieren dieser Eigenschaften, ist ein guter Ansatz, eine Referenzimplementierung zu entwickeln. Diese Referenz dient fortan als Beispiel. Weiterhin stellt sie das als erstes zu erreichende Ziel dar. Alle Bemühungen sollten darauf hinauslaufen, eine Applikation generieren zu lassen, welche die Referenzimplementierung gleicht.

Schon beim Entwickeln der Referenz muss sich der Entwickler Gedanken darüber machen, welche \textit{Views} die Anwendung besitzen soll. Diese \textit{Views} entscheiden auch über die Funktionalitäten, welche der Entwickler den Nutzern zur Verfügung stellen will. So wird bereits bei der Planung und Entwicklung der Applikation beschrieben welche Features realisiert werden. Dieser Funktionsumfang beschreibt ob der Nutzer Listen- und Detailansichten zur Verfügung hat und ob er Datensätze löschen, neu anlegen beziehungsweise bearbeiten darf. Mit der Entscheidung, dass es eine Möglichkeit zur Neuanlage und Bearbeitung von Datensätzen geben soll, muss zusätzlich festgelegt werden, welche Attribute des Datensatzes bearbeitet werden dürfen und welche minimal notwendig sind.

Ist bekannt welche \textit{Views} realisiert werden, muss über den Aufbau der einzelnen \textit{Views} entschieden werden. Es müssen Entscheidungen über die Anordnung der darzustellenden Informationen innerhalb einer \textit{View} getroffen werden. Diese Entscheidungen beinhalten neben der Strukturierung und Darstellung textueller Informationen auch Überlegungen zum Erscheinungsbild. Zum Erscheinungsbild gehören Eigenschaften wie Schriftgröße oder Schriftfarbe. Ist eine \textit{View} fertig designet, steht fest, in welcher Reihenfolge gegebene Informationen angezeigt werden. Ob Informationen wie Vorname und Nachname zusammengefasst werden. Auch hat der Entwickler entschieden ob alle existierenden Daten in der entsprechenden Ansicht relevant sind oder ob darauf verzichtet werden kann.  Auch ist klar wo und wie möglicherweise existierende Bilder dargestellt werden sollen.

Neben diesen auf das \textit{\acf{ui}} bezogenen Kriterien müssen auch Entscheidungen darüber gefällt werden, ob diese angezeigten Informationen ausschließlich informative Details sind oder ob diese interaktiv sind. Das heißt soll der Benutzer der Android Applikation die Möglichkeit haben weitere Funktionen durch das anklicken einer dieser Felder auszuführen. Mögliche Aktionen wären beispielsweise das Öffnen der Anwendung \textit{Maps} beim Klick auf eine Adresse oder das öffnen eines Webbrowsers beim anklicken eines Hyperlinks. 

All diese Entscheidungen, welche über den Aufbau und dem Design der Android Applikation entscheiden, sind für einen Generator wichtig. Dieser benötigt all diese Informationen um diese in der zu generierende Anwendung zu realisieren. Hierfür muss ein \textit{Meta-Modell} entwickelt werden, welches alle der oben genanten Beschreibungen im Bezug zur Android Applikation widerspiegelt. Das Modell muss alle Informationen über die Anzahl und Arten der \textit{Views}, deren Aufbau und die exakte Darstellung von Schrift, Bildern und möglichen Funktionen, welche bei Klick ausgeführt werden sollen besitzen.

\section{\textit{Meta-Modell}}
Nach dem die Referenzimplementierung vorstellt und analysiert  wurde, wurden alle relevanten Informationen erkannt und zusammen gestellt. Diese Zusammenstellung an Daten, welche die Applikation beschreiben wird \textit{Meta-Modell} genannt.

\subsection{Kompatibilität mit \acs{gemara} und andern möglichen Clients}
Da Enfield primär für die Generierung von Anwendungen im \textit{Backend}-Bereich entwickelt wurde, in welchem die Gestaltung von \acf{ui} eine eher untergeordnete Rolle spielen, muss die Erweiterung auch dieses Feature realisieren. Neben all diesen Erweiterungen muss auch sichergestellt werden, das das \textit{Meta-Modell} auch weiterhin für das generieren von \textit{Backends} genutzt werden kann. Idealerweise, ohne die Überarbeitung der bereits entwickelten Software-Generatoren. 

Die Abbildung \ref{fig:enfield-model} zeigt die vereinfachte Modell-Klasse des \textit{Enfield-Meta-Modells}. 
In dieser Klasse sind bereits die wichtigsten Informationen wie zum Beispiel der Name der Applikation oder unter welchem \textit{Package} diese zu finden ist vorhanden. Neben diesen grundsätzlichen Informationen liefert die Modell-Klasse auch den Startpunkt des \textit{endlichen Automaten}, welcher die Anwendung beschreibt. Dieser Startpunkt ist der \textit{GetDispatcherState}. Dieses Objekt besitzt das Attribut \textit{transitions}. Dieses Attribut beschreibt, welche \textit{States} auf den \textit{Dispatcher-State folgen}. Jeder dieser folgenden \textit{States}, besitzt wiederum eine Collection mit \textit{Transitionen}, welche auf die nachfolgenden \textit{States} verweisen. So wird mit Hilfe der \textit{Transitionen} und der \textit{States} der \textit{endliche Automat} beschrieben. Der Generator kann diese Beschreibung nutzen, um zu entscheiden in welcher Reihenfolge, welche Klassen generiert werden müssen.

\begin{figure}[H]
	\begin{center}
		\includegraphics[width=0.86\textwidth]{images/Enfield-Meta-Model.png}
		\caption{Vereinfachter Aufbau des \textit{Enfield-Meta-Modells}.}
		\label{fig:enfield-model}
	\end{center}
\end{figure}

Um jetzt zusätzlich benötigten Informationen für die Android Applikation in dieses bestehende Modell einzubauen, gibt es zwei Möglichkeiten.

\subsection{Eigenes \textit{Android-Meta-Modell}}

Es besteht die Möglichkeit die Modell-Klasse um ein Attribut \textit{Android-Meta-Modell} zu erweitern.
Die Abbildung \ref{fig:android-model} zeigt schemenhaft ein Beispiel wie ein \textit{Android-Meta-Modell} aussehen könnte. Auffällig hierbei ist, das viele Informationen, die das \textit{Enfield-Modell} bereits liefern würde, noch einmal explizit beschrieben werden müssen. Ein Beispiel wären die \textit{Transitionen}, zwischen den \textit{Fragmenten} beziehungsweise zwischen den \textit{Activities}. 


\begin{figure}[H]
	\begin{center}
		\includegraphics[width=0.86\textwidth]{images/Android-Meta-Model.png}
		\caption{Möglicher Aufbau eines \textit{Android-Meta-Modells}.}
		\label{fig:android-model}
	\end{center}
\end{figure}

Der Nutzer des Software-Generators, muss ziemlich viel über den Ablauf und die Funktionsweise einer Android-Anwendung wissen, um diesen Generator sinnvoll verwenden zu können.
Dabei bleibt zusätzlich noch die Möglichkeit, das der Nutzer eigens geschriebene Methoden in das Modell einpflegen kann. John Abou-Jaoudeh at al., haben in ihrer Arbeit \textit{A High-Level Modeling Language for Efficent Design, Implementation, and Testing of Android Applications}\cite{abou2015high} ein \textit{Meta-Modell} entwickelt, welches genau solche Features unterstützt.

Der Vorteil einer solchen Erweiterung des \textit{Enfield-Modell}s ist, das alle benötigten Daten für die Android Anwendung an einer Stelle zu finden sind. Auch hat der Nutzer die Möglichkeit an manchen Stellen eigene Methoden einzufügen und somit ist er in der Lage das Verhalten der App weiter zu individualisieren.

Jedoch überwiegen in diesem Fall die Nachteile. Ein Nachteil dieses Vorgehens ist, die redundante Beschreibung des Programm-Ablaufes. Einmal im \textit{Android-Meta-Modell} und einmal im \textit{Enfield-Meta-Modell}. Bei jeder Änderung gilt dies zu berücksichtigen. 
Der nächste Nachteil ist der, der Nutzer des Software-Generators muss sich in der Entwicklung von Android Anwendungen auskennen. Er muss genau das Zusammenspiel von \textit{ViewHoldern}, \textit{Adaptern}, \textit{Fragments} und \textit{Activities} kennen. Er muss wissen wie diese ineinandergreifen und wann welche Aktionen ausgelöst werden müssen. Weiterhin sollte er ein grundsätzliches Verständnis für das \textit{\acf{mvc} Pattern} besitzen, welches bei der Entwicklung von Android Applikationen Anwendung findet.
Ein weiterer Nachteil ist die Beschränkung des Modells auf Android. Wird das \textit{Enfield-Modell} um ein \textit{Android-Meta-Modell} erweitert, so muss dieses für jeden einzelnen \textit{Client} geschehen. Soll der Generator beispielsweise um Polymer-Webkomponente oder einer iOS-Anwendung erweitert werden, so müsste für jede einzelne Art von \textit{Client}, das \textit{Enfield-Modell} mit einem Entsprechenden \textit{Meta-Modell} erweitert werden.

\subsection{Allgemeine Erweiterungen des \textit{Enfield-Modells}}

In dieser Arbeit wurde sich für die Variante entschieden, das \textit{Enfield-Modell} an geeigneter Stelle zu erweitern.
Diese Stelle befindet sich in den einzelnen \textit{States}. Jede Instanz des \textit{AbstractState} besitzt ein Attribut \textit{SingleResourceView}. Diese Klasse wird um die  Attribute, welche benötigt werden erweitert. In der Abbildung \ref{fig:enfield-model-extended} ist der vereinfachte Aufbau des \textit{AbstractStates} und einer \textit{SingleResourceView} zu sehen.

Wird beispielsweise eine Instanz eines \textit{GetPrimarySingleResourceByIdStates} erzeugt, und dessen \textit{SingleResourceView} enthält alle notwendigen Informationen, um die \textit{View} in der Android Anwendung zu beschreiben. Kann der Generator mit Hilfe der \textit{Transitionen} über die \textit{States} iterieren und verfügt an jedem \textit{State} über alle benötigten Informationen, um den aktuellen \textit{State} in der Anwendung generieren zu lassen.

Bei dieser Methode befinden sich alle \textit{State}-spezifischen Daten direkt am \textit{State}. Jedoch gibt es neben diesen spezifischen Daten auch Daten, welche die komplette Applikation betreffen. Hierfür muss das \textit{Enfield-Modell} noch an einer andern Stelle erweitert werden. 
Es erscheint sinnvoll die Erweiterung direkt in der Modell-Klasse vorzunehmen. So kann der Generator schon am Anfang auf diese Daten zugreifen und diese verarbeiten.

Die Abbildung \ref{fig:enfield-model-extended} zeigt das \textit{Enfield-Modell}, welches um die oben genannten Informationen erweitert wurde.

\begin{figure}[H]
	\begin{center}
		\includegraphics[width=0.86\textwidth]{images/Enfield-Meta-Model-Erweitert.png}
		\caption{Vereinfachter Aufbau des erweiterten \textit{Enfield-Meta-Modells}.}
		\label{fig:enfield-model-extended}
	\end{center}
\end{figure}

Der Nachteil dieser Methode ist, das die Informationen an mehr als einer Stelle im \textit{Enfield-Modell} zu finden sind. Sollten die Informationen zu den \textit{Clients} verändert werden, so sind Änderungen an der \textit{SingleResourceView}-Klasse und in der Modell-Klasse nötig. Die Vorteile wurden jedoch oben schon einmal erwähnt. Der Generator kann das Modell als Fahrplan nutzen und weiß genau wann er welche Klassen für die Android Anwendung erzeugen muss. Er kann auch mit Hilfe der \textit{Transitionen} bestimmen wie der Verlauf innerhalb der Anwendung gestaltet ist.

\subsection{Analyse der benötigten Dateien für das \textit{Meta-Modell}}

Nachdem identifiziert wurde, an welchen Stellen das \textit{Enfield-Modell} erweitert werden soll, muss noch analysiert werden, welche Informationen an diesen Stellen zur Verfügung gestellt werden müssen. Bei dieser Analyse muss auch ein Augenmerk darauf gelegt werden, wie man die Informationen so aufbereitet, dass diese nicht nur eine Android-Applikation, sondern auch mögliche andere \textit{Clients} unterstützen.

Die Analyse in dieser Arbeit beschränken sich auf die \textit{Clients} Android und Polymer-Webkomponente. Bei beiden wird das \acf{ui} nach den Richtlinien,des von Google entwickelten Material Design, erstellt \cite{material}. Diese Richtlinien schreiben bereits viele nötigen Informationen für die Oberflächengestaltung vor. So wird beispielsweise definiert, das Einträge in einer Liste, als Karte dargestellt werden sollen. Abstände und Icons werden ebenfalls festgelegt.

\subsubsection{CardView}

\begin{figure}[H]
	\begin{center}
		\includegraphics[width=0.86\textwidth]{images/card.png}
		\caption{Beispiel einer \textit{CardView} aus einer Liste von Dozenten nach Material Design.}
		\label{fig:card}
	\end{center}
\end{figure}

\newpage

\begin{lstlisting}[label=lst:braun_json,
language=json,
firstnumber=1,
caption=Demo Daten eines Dozenten.]	
...			   
{
"address": "Sanderheinrichsleitenweg 20 97074 Wuerzburg",
"chargeUrl": {
"href": "https://apistaging.fiw.fhws.de/mig/api/lecturers/4/charges",
"rel": "chargeUrl",
"type": "application/vnd.fhws-charge.default+json"
},
"email": "peter.braun@fhws.de",
"firstName": "Peter",
"homepage": {
"href": "http://www.welearn.de/.../prof-dr-peter-braun.html",
"rel": "homepage",
"type": "text/html"
},
"id": 4,
"lastName": "Braun",
"phone": "0931/3511-8971",
"profileImageUrl": {
"href":"https://apistaging.fiw.fhws.de/.../4/profileimage",
"rel": "profileImageUrl",
"type": "image/png"
},
"roomNumber": "I.3.27",
"self": {
"href": "https://apistaging.fiw.fhws.de/mig/api/lecturers/4",
"rel": "self",
"type": "application/vnd.fhws-lecturer.default+json"
},
"title": "Prof. Dr."
}
...
\end{lstlisting}

Die  \textit{\acf{json}} Repräsentation unter Listing \ref{lst:braun_json} beschreibt das Beispiel aus Abbildung \ref{fig:card}.
Jetzt gilt es zu überlegen, wie die Attribute des \ac{json} Objekts aufzubereiten sind, dass diese die Karte des Dozenten widerspiegeln. 
In erster Linie muss entschieden werden, welche der gelieferten Informationen sollen in der Liste für jeden einzelnen Dozenten angezeigt werden. Ist es sinnvoll Informationen zu gruppieren? Hier beispielsweise die Attribute \textit{firstName} und \textit{lastName}, diese sollen in einer Zeile angezeigt werden. Ist bekannt welche Informationen eine Karte enthalten soll, so muss auch noch die Reihenfolge der einzelnen Attribute auf Karte bestimmt werden.
Neben der Reihenfolge gibt es noch die Möglichkeit Schriftgröße oder Schriftfarbe der einzelnen Attribute unterschiedlich zu gestalten. Auch müssen die Standardicons den jeweiligen Attributen zuweisen werden. Es sollte weitergehend möglich sein einzelnen Attribute bestimmte Aktionen zuzuweisen. Beispielsweise beim Klick auf eine Homepage, sollte diese im Browser geöffnet werden, oder beim Klick auf die Adresse sollte die Applikation \textit{Maps} öffnen und die angeklickte Adresse dort anzeigen. Ein Attribut mit dem Hyperlink zu einer Website, sollte es möglich sein einen mitgegebenen Text anstelle des Hyperlinks anzuzeigen. 

Besitzt die Karte ein Bild, so sollte der Nutzer die Möglichkeit besitzen zu entscheiden ob dieses auf der linken oder rechten Seite der Karte dargestellt werden soll.

\subsubsection{\textit{DetailView}}

\begin{figure}[H]
	\begin{center}
		\includegraphics[width=0.4\textwidth]{images/detail.png}
		\caption{Beispiel einer \textit{DetailView} eines Dozenten nach Material Design.}
		\label{fig:detail}
	\end{center}
\end{figure}

Die zur Verfügung stehenden Daten sind die gleichen, welche unter Listing \ref{lst:braun_json} einzusehen sind.

Analog wie bei der \textit{CardView} stellt sich auch bei der \textit{DetailView} die Frage, welche Daten dargestellt werden sollen. Hier jedoch gibt es zusätzlich zu der horizontalen Gruppierung (Beispiel: Vornamen und Nachnamen), auch noch eine vertikale Gruppierung. Diese wird im weiteren auch Kategorisierung genannt. In der detaillierten Ansicht eines Dozenten gibt es die Möglichkeit Attribute zu kategorisieren und jeder Kategorie mit einem Namen zu versehen. Für die Gestaltung und Anordnung sowie mögliche Klick-Aktionen müssen die selben Anforderungen wie bei der \textit{CardView} berücksichtigt werden. 

Jedoch muss die \textit{DetailView} wissen, welches Attribut den Titel der \textit{View} darstellt, da dieser in der \textit{AppBar} erscheinen wird. In diesem Beispiel ist es der Name des Dozenten. Anders als bei der \textit{CardView} gibt es hier nicht die Möglichkeit zu bestimmen wo das Bild dargestellt werden soll. Ist ein Bild vorhanden, so wird dieses in der \textit{CollapsingToolbar} dargestellt \cite{collapsing}.
subsubsection{InputView}

\begin{figure}[H]
	\begin{center}
		\includegraphics[width=0.4\textwidth]{images/input.png}
		\caption{Beispiel einer \textit{View} zum Anlegen eines Dozenten.}
		\label{fig:input}
	\end{center}
\end{figure}

Für das neu Anlegen eines Dozenten oder auch zum bearbeiten muss entschieden werden, welche Attribute zum Anlegen nötig sind. Auch hier ist es notwendig die Reihenfolge zu bestimmen. Jedoch kommen in dieser \textit{} für jedes Attribut noch die Möglichkeit hinzu ein \textit{Hint}-Text anzugeben. Dieser Text beschreibt, was in der Android \textit{View} \textit{EditText} als Beschreibung für das bestimmte Attribut steht. Weiter sollte es die Möglichkeit geben, jedem Feld eine Nachricht mitzugeben, welche angezeigt wird, wenn das Feld beispielsweise leer gelassen wird. Oder eine weitere Nachricht, wenn das Eingegebene nicht dem Erwarteten entspricht. Zum Beispiel wurde in das Feld für die E-Mail eine Telefonnummer eingegeben. Oder es wurde ein regulärer Ausdruck mitgegeben und das Eingegebene entspricht nicht dessen Anforderungen.

\subsubsection{Programmablauf und Klick-Aktionen}

Da das \textit{Enfield-Modell} bereits einen \textit{endlichen Automaten} beschreibt, welcher den Programmablauf widerspiegelt, ist es nicht notwendig, diesen Ablauf noch einmal genauer zu definieren. Der bereits definierte Ablauf übernommen wird.

Auch die Aktionen welche durch einen Klick auf ein bestimmtes Attribut ausgeführt werden soll, beschränkt sich auf Android Standardaktionen. Beispielsweise das wechseln zu den \textit{Maps}, zu einem \textit{E-Mail Client}, dem \textit{Browser} oder zum \textit{Anrufsmenü}. Jede dieser Aktion ergibt sich aus den Typen der Attribute, weswegen diese auch nicht weiter definiert werden müssen.

\subsection{Design der \textit{View-Meta-Modelle}} \label{sec:resourceViews}

In den letzten Abschnitten der Arbeit wurde aufgezählt, was das \textit{Meta-Modell} sowohl Android- als auch Polymer-seitig abdecken muss. In diesem Kapitel wird ein \textit{Meta-Modell} vorgestellt, welches die erwähnten Eigenschaften abdeckt.


\begin{figure}[H]
	\begin{center}
		\includegraphics[width=\textwidth]{images/metamodel.png}
		\caption{Aufbau der \textit{Views} zur Erweiterung des \textit{Enfield-Modell}s.}
		\label{fig:meta-model}
	\end{center}
\end{figure}

Die Abbildung \ref{fig:meta-model} zeigt den Aufbau der Objekte, mit welchem das \textit{Enfield-Modell} erweitert wird. Die drei \textit{Views}: \textit{CardView}, \textit{DetailView} und \textit{InputView} sind alles Instanzen von \textit{AbstraktResourceView}. Jede der \textit{View}, weiß, durch die Zuordnung mit Hilfe des Ressourcennamens, welche Ressource sie darstellen soll. Die drei \textit{Views}, lassen sich in zwei Kategorien einteilen: \textit{Views}, welche Informationen anzeigen und \textit{Views} welche zur Eingabe von Informationen benötigt werden.
So gehören \textit{CardView} und \textit{DetailView} zu den anzeigenden \textit{Views} und die \textit{InputView} zur zweiten Kategorie. 

\subsubsection{Anzeigende \textit{Views}}
Diese \textit{View}-Typen haben die Aufgabe eine Liste aller Attribute zu halten, welche in der entsprechenden \textit{View} angezeigt werden sollen. Dabei bestimmt die Reihenfolge, in welcher die Attribute in dieser Liste sind auch die Anordnung in der Oberfläche. Ist das erste Item in der Liste der Name, so wird dieser ganz oben in der \textit{View} angezeigt.
Bei der \textit{DetailView} jedoch gibt es nicht eine Liste mit den Attributen, sondern eine Liste mit Kategorien. Diese besitzen 
einen Namen und eine Liste mit den Attributen ihrer Kategorie. Die Darstellungsreihenfolge der Kategorien und deren Attribute ist analog zu der der \textit{CardView}. Weiter besitzt die \textit{DetailView} das Attribut \textit{image}, dieses wird hier aus der Liste der Attribute herausgezogen, da dieses Attribut bestimmt, ob die \textit{View} eine \textit{CollapsingToolbar} besitzen wird oder nicht. Wiederum haben beide \textit{Views} das Attribut \textit{titleOfResource} dieses bestimmt welches Attribut unserer Ressource beispielsweise in der \textit{Toolbar} angezeigt wird.

Auf die Polymer-spezifischen Attribute wird in dieser Arbeit nicht weiter eingegangen.

Mit Hilfe der Listen, Titelattributen und dem Bildattribut kann das Erscheinungsbild einer \textit{View} schon ziemlich gut beschrieben werden. Als nächstes wird auf Möglichkeit, Schriftgrößen, Schriftfarben und Klick-Aktionen zu definieren.
Außerdem ist es bis jetzt nur möglich einfache Attribute anzuzeigen, eine horizontale Gruppierung ist noch nicht möglich. Um diese Anforderungen zu erfüllen, werden nicht Attribute in den Listen gespeichert sondern Ausprägungen von \textit{ResourceViewAttributen}. 

Es gibt zwei Ausprägungsarten: \textit{SingleResourceViewAttribute} und \textit{GroupedResourceViewAttribute}.  Das \textit{SingleResourceViewAttribute} ist für einfache Attribute, mit diesem ist es beispielsweise möglich den Titel eines Dozenten anzuzeigen. Das \textit{GroupedresourceViewAttribute} ermöglicht die horizontale Gruppierung. Beide Objekte, bestimmen jedoch nicht die Design-spezifischen Eigenschaften des Attributs. Hierfür besitzen beide Attribut-Typen das Attribut \textit{DisplayViewAttribute}.

Bei der \textit{SingleResourceViewAttribute} ist diese Instanz von einem \textit{AbstractViewAttribute} das einzige Attribut, beim \textit{GroupedResourceViewAttribute} wiederum gibt es eine Liste von diesen \textit{DisplayViewAttributen}, welche dann die anzuzeigenden Informationen widerspiegeln. Weitergehend besitzt das \textit{GroupedResourceViewAttribute} auch noch ein \textit{DisplayViewAttribute}, welches die neu entstandene Gruppierung beschreiben soll.

Ein \textit{DisplayViewAttribute} besitzt nun die Möglichkeit, Schriftgröße und -farbe zu definieren. Die angegebene Farbe muss eine in hexadezimaler Darstellung angegeben werden, wird keine Farbe mitgegeben, wird die Defaultfarbe der Anwendung genommen. In der Regel ist diese Schwarz.  Die Schriftgröße wiederum ist auf 3 Stufen beschränkt. Es gibt die Möglichkeit den Text in klein, normal und groß darzustellen. Per default ist normal eingestellt. Aus der Oberklasse \textit{AbstractViewAttribute} besitzt das \textit{DisplayViewAttribute} noch die Attribute \textit{attributeName}, dieses muss exakt so heißen wie in der Definition der Ressource.
Mit dem \textit{attributeLabel} kann angegeben werden, wie dieses Attribut in der \textit{View} angezeigt werden soll. Die Abbildung \ref{fig:detail} zeigt die Verwendung von den Labels, vor beispielsweise der E-Mailadresse des Dozenten steht \textit{E-Mail}, dieser String entspricht dem Label des Attributes. Weiterhin muss angeben werden von welchem Typ das aktuell beschriebene Attribut ist.
Dies geschickt mit dem Attribut \textit{AttributeType}. Es gibt folgende mögliche Typen: \textit{HOME}, \textit{MAIL}, \textit{LOCATION}, \textit{PICTURE}, \textit{PHONE\_NUMBER}, \textit{TEXT}, \textit{URL}, \textit{DATE}, \textit{SUBRESOURCE}. Jeder Typ bestimmt die Eigenschaften des Attributes. Über diesen wird bestimmt welches Icon in der Karte vor dem entsprechenden Attribut angezeigt wird oder welche Aktion bei Klick ausgeführt werden soll. So wird bei einem Klick auf ein Attribut vom Typ \textit{LOCATION} versucht die Anwendung \textit{Maps} zu öffnen und den angezeigten Standort dort anzuzeigen. Ist das Attribute vom Typ \textit{SUBRESOURCE} so wird für dieses Attribut ein \textit{Button} angezeigt, dieser ermöglicht es dann zu der entsprechenden Subressource zu wechseln. Diese Klick-Aktionen müssen jedoch mit dem Attribut \textit{clickActionAndroid} erst aktiviert werden.
Manche Typen bringen noch ein paar andere Besonderheiten mit sich. So muss man beispielsweise bei einem \textit{URL}-Attribut noch eine Beschreibung mitgeben, welche anstelle der Hyperlinks angezeigt werden soll. Bei einem Bild kann beispielsweise noch bestimmt werden, ob dieses links oder rechts dargestellt werden soll. 

Nachfolgend wird auf einige Besonderheiten der Nutzung der eingebenden \textit{Views} eingegangen und diese genauer erklärt. So zeigt Listing \ref{lst:grouped} beispielsweise das Erzeugen eines \textit{GroupedResourceViewAttributes}. Hierfür werden erst einmal drei \textit{DisplayViewAttribute} definiert. Das erste beschreibt hierbei das Aussehen, den Namen und den Typ der Gruppierung. Die Gruppe in diesem Beispiel wird aus den beiden Attributen \textit{firstName} und \textit{lastName} zusammengesetzt. Beide Attribute sind vom Typ \textit{TEXT} auch die Gruppe wird von diesem Typ sein. Der String im Konstruktor ist der Name dieses Attributes. Stellt das Attribut ein Attribut aus der Ressource dar, wie der \textit{firstName} beziehungsweise \textit{lastName} so muss dieser Name identisch mit dem Attribut der Ressource sein. Neben der Zusammensetzung der Gruppe, wird hier ebenfalls definiert, wie dies dargestellt werden soll. Mit \textit{setFontSize(DisplayViewAttribute.FontSize.LARGE)} wird deklariert das die Gruppe mit einer großen Schriftgröße dargestellt werden, und die Methode \textit{setFontColor("\#000")} bestimmt das die Schrift schwarz ist. Bei einer Gruppe hat es kein Effekt wenn die Schriftfarbe oder Größe der einzelnen Gruppenmitglieder bestimmt wird. Die Darstellung ist einzig von den Attributen der Gruppe abhängig.

\begin{lstlisting}[label=lst:grouped,
language=java,
firstnumber=1,
caption=Erstellung eines \textit{GroupedResourceViewAttributes}.]		
DisplayViewAttribute nameAttribute = new DisplayViewAttribute("name", ViewAttribute.AttributeType.TEXT);
nameAttribute.setFontSize(DisplayViewAttribute.FontSize.LARGE);
List<DisplayViewAttribute> nameAttributes = new ArrayList<>();

DisplayViewAttribute firstNameAttributes = new DisplayViewAttribute("firstName", ViewAttribute.AttributeType.TEXT);
firstNameAttributes.setAttributeLabel("FirstName");
nameAttributes.add(firstNameAttributes);

DisplayViewAttribute lastNameAttributes = new DisplayViewAttribute("lastName", ViewAttribute.AttributeType.TEXT);
lastNameAttributes.setAttributeLabel("LastName");
nameAttributes.add(lastNameAttributes);

GroupResourceViewAttribute name;
try {
nameAttribute.setFontColor("#000");
name = new GroupResourceViewAttribute(nameAttribute, nameAttributes);
} catch (DisplayViewException ex) {
name = null;
}
\end{lstlisting}

Das Listing \ref{lst:category} beschreibt die Definition einer \textit{Category} als Teil einer \textit{DetailView}. In diesem Listing wird eine \textit{Category} mit dem Namen \textit{Office} erzeugt. Diese Kategorie besitzt zwei Attribute, welche als \textit{DisplayViewAttribute} dargestellt werden. Eines der beiden Attribute ist in diesem Fall die Adresse. Es wird definiert, das dieses \textit{DisplayViewAttribute} vom Type \textit{LOCATION} ist und das es eine Aktion beim anklicken geben soll. Des weiteren wird definiert, das dieses Attribut ein Label \textit{Address} besitzt.

\begin{lstlisting}[label=lst:category,
language=java,
firstnumber=1,
caption=Erstellung einer \textit{Category}.]		
...
new Category("Office", getOfficeResourceViewAttributes());
...

private static List<ResourceViewAttribute> getOfficeResourceViewAttributes() {
List<ResourceViewAttribute> officeAttributes = new ArrayList<>();

DisplayViewAttribute addressAttribute = new DisplayViewAttribute("address", ViewAttribute.AttributeType.LOCATION);
addressAttribute.setAttributeLabel("Address");
addressAttribute.setClickActionAndroid(true);
SingleResourceViewAttribute address = new SingleResourceViewAttribute(addressAttribute);
officeAttributes.add(address);

DisplayViewAttribute roomAttribute = new DisplayViewAttribute("roomNumber", ViewAttribute.AttributeType.TEXT);
roomAttribute.setAttributeLabel("Room");
roomAttribute.setClickActionAndroid(true);
SingleResourceViewAttribute room = new SingleResourceViewAttribute(roomAttribute);
officeAttributes.add(room);

return officeAttributes;
}
\end{lstlisting}

Im Anhang befinden sich unter Listing \ref{lst:detailview_impl} die vollständige Definition einer \textit{DetailView}, sowie unter Listing \ref{lst:cardview_impl} die vollständige Definition einer \textit{CardView}. 

\subsubsection{Eingebende \textit{Views}}

Bei der \textit{InputView} gibt es wieder eine Liste, welche dieses mal \textit{InputViewAttribute} mit der Oberklasse \textit{AbstractViewAttribute} hält. Diese Liste bestimmt analog zu den anzeigenden Views die darzustellende Reihenfolge der Attribute. 

Neben dem \textit{attributeName} der wieder exakt dem Namen aus der Ressourcendefinition entsprechen muss, besitzt das \textit{InputViewAttribute} auch die Möglichkeit zu bestimmen, welcher Typ das aktuelle Attribut besitzt. Jedoch haben die Typen hier eine andere Bedeutung als bei dem anderen \textit{View}-Typ. So wird beispielsweise bei dem Type \textit{DATE} kein \textit{EditText} angezeigt, sondern der Nutzer hat die Möglichkeit das Datum über das \textit{DatePicker-Widget} von Android einzugeben. 

Es ist jedoch für den Android-\textit{Client} nicht möglich Bilder zu Ressourcen hinzuzufügen, oder diese zu Bearbeiten. Des weiteren wird eine Subressource nicht in einer \textit{InputView} der Oberressource bearbeitet oder neu angelegt. Dies geschieht in der entsprechenden \textit{View} der Subressource. Die anderen Typen beschränken das \textit{EditText-Widget} auf die angegebenen Typen. So wird beispielsweise bei einem Klick auf ein \textit{PHONE\_NUMBER-Feld} die Tastatur im Zahlenmodus ausgefahren.

Einem \textit{InputViewAttribute} muss zusätzlich ein \textit{hintText} mitgegeben werden, der im \textit{EditText} des Attributs beschreibt, was in diesem Feld erwartet wird. Mit dem String \textit{missingText} kann dem Attribut mitgegeben werden, welche Nachricht dem Nutzer angezeigt wird, falls er versucht zu speichern ohne das entsprechende Feld auszufüllen. Mit der Kombination von \textit{checkPattern} und \textit{errorText} bekommt der Nutzer des Generators die Möglichkeit die Validierung des eingegebenen Attributes noch weiter zu verfeinern und auch dem Nutzer der Applikation ein Feedback zu geben, falls eine falsche Eingabe getätigt wurde.

Das Listing \ref{lst:input} stellt dar, wie ein \textit{InputViewAttribute} für eine \textit{InputView} definiert werden muss. Das hier initialisierte Attribut ist vom Type \textit{TEXT} und wird mit dem \textit{attributeName} \textit{roomNumber} seinem zugehörigen Attribut der Ressource zugewiesen.
Auch wird hier wieder ein Label vergeben, daneben den \textit{Hint-Text} \textit{Room} sowie der \textit{Missing-Text} \textit{Room is missing!}.
Im Anahng unter Listing \ref{lst:inputview_impl} befindet sich eine vollständige Definition einer \textit{InputView}.

\begin{lstlisting}[label=lst:input,
language=java,
firstnumber=1,
caption=Definition von \textit{InputViewAttributes} einer \textit{InputView}.]	
InputViewAttribute room = new InputViewAttribute(
"roomNumber", ViewAttribute.AttributeType.TEXT, 
"Room", "Room is missing!");
room.setAttributeLabel("Room");
inputViewAttributes.add(room);
\end{lstlisting}

\subsection{Analyse und Design von allgemeinen Daten für eine Anwendung}

Dieses Kapitel behandelt die Informationen, welche eine Applikation neben den \textit{View}-Beschreibungen zusätzlich benötigt, aber diese vom Kontext her nicht in einer der \textit{Views} beschrieben werden können.

Ein Beispiel für eine solche Information wäre der \acf{url} für den Einstieg. Die Applikation benötigt diesen um zu wissen, unter welcher Adresse die anzuzeigenden Informationen zu finden sind. Ein weiteres Beispiel sind die Grundfarben der Applikation. Das Material Design gibt drei benötigte Grundfarben vor: \textit{colorPrimary}, \textit{colorprimaryDark} und \textit{colorAccent} diese Grundfarben werden um die Farbe für den \textit{Toolbar}-Text erweitert.

Mit dem Wissen, konnte eine Erweiterung des \textit{Enfield-Modell}s designet werden, welches in Abbildung \ref{fig:appspecifics} dargestellt ist.

\begin{figure}[H]
	\begin{center}
		\includegraphics[width=0.86\textwidth]{images/appspecifics.png}
		\caption{Aufbau des \textit{AppSpecifics} Objekt  zur Erweiterung des \textit{Enfield-Modell}s.}
		\label{fig:appspecifics}
	\end{center}
\end{figure}

Über die \textit{Map} \textit{additionalInformation} können zusätzlich weitere allgemeine Informationen an den Generator, weitergegeben werden.


\chapter{Lösung}
Dieses Kapitel befasst sich mit den Möglichkeiten und Lösungsansätze, zu den Problemstellungen aus Kapitel \ref{ch:problem}. Anhand von Beispielen wird verdeutlicht, wie gewisse Anforderungen umgesetzt werden könnten und wurden.

\section{Meta-Model}
Nach der Anforderungsanalyse, wurden alle relevanten Informationen erkannt und zusammen gestellt. Diese Zusammenstellung an Daten, welche die Applikation beschreiben wird Meta-Model genannt.

\subsection{Kompatibilität mit \acs{gemara} und andern möglichen Clients}

Um die Kompatibilität mit \acf{gemara} zu waren, wurde das Enfield-Meta-Model untersucht. Mit Hilfe dieser Untersuchung konnte festgestellt werden, an welcher Stelle zusätzliche Informationen für die Clients am sinnvollsten eingebaut werden können. So das diese den Ablauf der Applikation gut beschreiben und an den benötigten Stellen alle relevanten Informationen für den Software-Generator zur Verfügung stellt.

Die Abbildung \ref{fig:enfield-model} zeigt die vereinfachte Model-Klasse des Enfield-Meta-Models. 
In dieser Klasse sind bereits die wichtigsten Informationen wie zum Beispiel der Name der Applikation, oder unter welchem Package diese zu finden ist. Neben diesen grundsätzlichen Informationen liefert die Model-Klasse auch den Startpunkt des endlichen Automaten, welcher die Anwendung beschreibt. Dieser Startpunkt ist der \textit{GetDispatcherState}. Dieses Objekt besitzt das Attribut \textit{transitions}. Dieses Attribut beschreibt, welche States auf den Dispatcher-State folgen können. Jeder dieser folgenden States, besitzt wiederum eine Collection mit Transitionen, die auf die nachfolgenden States verweisen. So wird mit Hilfe der Transitionen und der States der endliche Automat der Anwendung beschrieben. Der Generator kann diese Beschreibung nutzen, um zu entscheiden in welcher Reihenfolge, welche Klassen generiert werden müssen.

\begin{figure}[H]
	\begin{center}
		\includegraphics[width=0.86\textwidth]{images/Enfield-Meta-Model.png}
		\caption{Vereinfachter Aufbau des Enfield-Meta-Models}
		\label{fig:enfield-model}
	\end{center}
\end{figure}

Um jetzt zusätzlich benötigten Informationen für die Android Applikation in dieses bestehende Modell einzubauen, gibt es zwei Möglichkeiten.

\subsection{Eigenes Android-Meta-Model}

Es besteht die Möglichkeit die Model-Klasse um ein Attribut \textit{Android-Meta-Model} zu erweitern.
Die Abbildung \ref{fig:android-model} stellt zeigt schemenhaft ein Beispiel wie ein Android-Meta-Model aussehen könnte. Auffällig hierbei ist das viele Informationen, die das Enfield-Model bereits liefern würde, hier noch einmal explizit beschrieben werden muss. Ein Beispiel wären die Transitionen, zwischen den Fragmenten beziehungsweise zwischen den Activities. 


\begin{figure}[H]
	\begin{center}
		\includegraphics[width=0.86\textwidth]{images/Android-Meta-Model.png}
		\caption{Möglicher Aufbau eines Android-Meta-Models}
		\label{fig:android-model}
	\end{center}
\end{figure}

Der Nutzer des Software-Generators, muss also ziemlich viel über den Ablauf und die Funktionsweiße einer Android-Anwendung wissen, um diesen Generator sinnvoll verwenden zu können.
Dabei bleibt zusätzlich noch die Möglichkeit, das der Nutzer eigens geschriebene Metohden in das Model einpflegen kann. John Abou-Jaoudeh at al., haben in ihrer Arbeit \textit{A High-Level Modeling Language for Efficent Design, Implementation, and Testing of Android Applications} ein Meta-Model entwickelt, welches genau solche Features unterstützt \cite{abou2015high}.

Der Vorteil einer solchen Erweiterung des Enfield-Models ist, das alle benötigten Daten für die Android Anwendung an einer Stelle zu finden sind. Auch hat der Nutzer die Möglichkeit an manchen Stellen eigene Methoden einzufügen und somit ist er in der Lage das Verhalten der App weiter zu individualisieren.

Jedoch überwiegen in diesem Fall die Nachteile. Ein Nachteil dieses Vorgehens ist, die redundante Beschreibung des Programm-Ablaufes. Einmal im Android-Meta-Model und einmal im Enfield-Meta-Model. Bei jeder Änderung gilt dies zu berücksichtigen. 
Der nächste Nachteil ist der Nutzer des muss sich in der Entwicklung von Android Anwendungen auskennen. Er muss genau das Zusammenspiel von ViewHoldern, Adaptern, Fragments und Activities kennen. Er muss wissen wie diese ineinandergreifen und wann welche Aktionen ausgelöst werden müssen. Weiterhin sollte er ein Grundsätzliches Verständnis für das \acf{mvc} Pattern besitzen, welches bei der Entwicklung von Android Applikationen anwendung findet.
Ein weiterer Nachteil ist die Beschränkung des Models auf Android. Wird das Enfield-Model um ein Android-Meta-Model erweitert, so muss dieses für jeden einzelnen Client geschehen. Soll der Generator beispielsweise um Polymer-Webkomponente oder einer iOS-Anwendung erweitert werden, so müsste für jede einzelne Art von Client, das Enfield-Model mit einem Entsprechenden Meta-Model erweitert werden.

\subsection{Allgemeine Erweiterungen des Enfield-Models an entsprechender Stelle}

In dieser Arbeit wurde sich für die Variante entschieden, das Enfield-Model an ihren \textit{SingleResourceViews} zu erweitern.
Wird diese Klasse um die  Attribute, die benötigt werden Attribute erweitert, so erhält der Generator die benötigten Ressourcen immer zur rechten Zeit.

Wird beispielsweise eine Instanz eines \textit{GetPrimarySingleResourceByIdStates} erzeugt, und dessen \textit{SingleResourceView} enthält alle notwendigen Informationen, um die View in der Android Anwendung zu beschreiben. Kann der Generator mit Hilfe der Transitionen über die States iterieren und verfügt an jedem State über alle benötigten Informationen, um den aktuellen State in der Anwendung generieren zu lassen.

Bei dieser Methode befinden sich alle State-spezifischen Daten direkt am State. Jedoch gibt es neben diesen spezifischen Daten auch Daten, welche die komplette Applikation betreffen, muss das Enfield-Model noch an einer andern Stelle erweitert werden. 
Hierfür erscheint es sinnvoll die Erweiterung direkt in der Model-Klasse vorzunehmen. So kann der Generator schon am Anfang auf diese Daten zugreifen und diese verarbeiten.

Die Abbildung \ref{fig:enfield-model-extended} zeigt das Enfield-Model, welches um die oben genannten Informationen erweitert wurden.

\begin{figure}[H]
	\begin{center}
		\includegraphics[width=0.86\textwidth]{images/Enfield-Meta-Model-Erweitert.png}
		\caption{Vereinfachter Aufbau des erweiterten Enfield-Meta-Models}
		\label{fig:enfield-model-extended}
	\end{center}
\end{figure}

Der Nachteil dieser Methode ist, das die Informationen an mehr als einer Stelle im Enfield-Modell zu finden sind. Sollten die Informationen zu den Clients verändert werden, so sind Änderungen an der SingleResourceView-Klasse und in der Model-Klasse nötig. Die Vorteile wurden jedoch oben schon einmal erwähnt. Der Generator kann das Model als Fahrplan nutzen und weiß genau wann er welche Klassen für die Android Anwendung erzeugen muss. Er kann auch mit Hilfe der Transitionen bestimmen wie der Verlauf innerhalb der Anwendung ablaufen soll.

\newpage
\subsection{Analyse der benötigten Dateien für das Meta-Model}

Nachdem identifiziert wurde, an welchen Stellen das Enfield-Model erweitert werden soll, muss noch analysiert werden, welche Informationen an diesen Stellen zur Verfügung stehen müssen. Bei dieser Analyse muss auch ein Augenmerk darauf gelegt werden, wie man die Informationen so aufbereitet, dass diese nicht nur eine Android-Applikation, sondern mögliche andere Clients unterstützen.

Die Analyse in dieser Arbeit beschränken sich auf die Clients Android und Polymere-Webkomponente. Bei beiden wird das \acf{ui} nach den Guidelines,des von Google entwickelten Material Design, erstellt \cite{material}. Diese Guidelines schreiben bereits viele nötigen Informationen für die Oberflächengestaltung vor. So wird beispielsweise definiert, das Einträge in einer Liste, als Karte dargestellt werden sollen. Abstände und Icons werden ebenfalls festgelegt.

\subsubsection{CardView}

\begin{figure}[H]
	\begin{center}
		\includegraphics[width=0.86\textwidth]{images/card.png}
		\caption{Beispiel einer CardView aus einer Liste von Dozenten nach Material Design Guidelines.}
		\label{fig:card}
	\end{center}
\end{figure}

\newpage

\begin{lstlisting}[label=lst:braun_json,
language=json,
firstnumber=1,
caption=Demo Daten eines Dozenten.]	
...			   
{
	"address": "Sanderheinrichsleitenweg 20 97074 Wuerzburg",
	"chargeUrl": {
		"href": "https://apistaging.fiw.fhws.de/mig/api/lecturers/4/charges",
		"rel": "chargeUrl",
		"type": "application/vnd.fhws-charge.default+json"
	},
	"email": "peter.braun@fhws.de",
	"firstName": "Peter",
	"homepage": {
		"href": "http://www.welearn.de/.../prof-dr-peter-braun.html",
		"rel": "homepage",
		"type": "text/html"
	},
	"id": 4,
	"lastName": "Braun",
	"phone": "0931/3511-8971",
	"profileImageUrl": {
		"href":"https://apistaging.fiw.fhws.de/.../4/profileimage",
		"rel": "profileImageUrl",
		"type": "image/png"
	},
	"roomNumber": "I.3.27",
	"self": {
		"href": "https://apistaging.fiw.fhws.de/mig/api/lecturers/4",
		"rel": "self",
		"type": "application/vnd.fhws-lecturer.default+json"
	},
	"title": "Prof. Dr."
}
...
\end{lstlisting}

Die  \acf{json} Repräsentation unter Listing \ref{lst:braun_json} beschreibt das Beispiel aus Abbildung \ref{fig:card}.
Jetzt gilt es zu überlegen, wie die Attribute des \ac{json} Objekts aufzubereiten sind, dass diese die Karte des Dozenten widerspiegeln. 

In erster Linie muss entschieden werden, welche der gelieferten Informationen sollen in der Liste für jeden einzelnen Dozenten angezeigt werden. Ist es sinnvoll ist Informationen zu gruppieren? Hier beispielsweise die Attribute \textit{firstName} und \textit{lastName}, diese sollen in einer Zeile angezeigt werden. Ist bekannt welche Informationen eine Karte enthalten soll, so muss auch noch die Reihenfolge der einzelnen Attribute der Karte bestimmt werden.
Neben der Reihenfolge gibt es noch die Möglichkeit das die Schriftgröße oder die Schriftfarbe der einzelnen Attribute unterschiedlich sein können. Auch müssen die Standardicons den einzelnen Attribute zuweisen werden. Auch sollte es die möglich sein einzelnen Attribute bestimmte Aktionen zuzuweisen. So sollte beispielsweise beim Klick auf eine Homepage auch diese im Browser geöffnet werden, oder beim Klick auf die Adresse sollte die Applikation Maps öffnen und die angeklickte Adresse dort anzeigen. Gibt es ein Attribut mit dem Hyperlink zu einer Website, sollte es möglich sein einen Text anzugeben, der anstelle des Hyperlinks angezeigt wird. 

Besitzt die Karte ein Bild, so sollte der Nutzer die Möglichkeit besitzen zu entscheiden ob er dieses gerne auf der linken oder der rechten Seite der Karte haben möchte.

\subsubsection{DetailView}

\begin{figure}[H]
	\begin{center}
		\includegraphics[width=0.4\textwidth]{images/detail.png}
		\caption{Beispiel einer DetailView eines Dozenten nach Material Design Guidelines}
		\label{fig:detail}
	\end{center}
\end{figure}

Die zur Verfügung  stehenden Daten sind die gleichen, welche unter Listing \ref{lst:braun_json} einzusehen sind.

Analog wie bei der CardView stellen Sich auch bei der DetailView, welche Daten alle dargestellt werden sollen. Hier jedoch gibt es zusätzlich zu der horizontalen Gruppierung (Beispiel mit den Vornamen und Nachnamen), auch noch eine vertikale Gruppierung, die im weiteren auch Kategorisierung genannt wird. In der detaillierten Ansicht eines Dozenten gibt es die Möglichkeit die Attribute zu kategorisieren und jeder Kategorie mit einem Namen zu versehen. Für die Gestaltung und Anordnung sowie mögliche Klick-Aktionen müssen die selben Anforderungen wie bei der CardView berücksichtigt werden. 

Jedoch muss die DetailView wissen, welches Attribut den Titel der View darstellt, da dieser in der AppBar erscheinen wird. In diesem Beispiel ist es der Name des Dozenten. Anders als bei der CardView gibt es hier nicht die Möglichkeit zu bestimmen wo das Bild dargestellt werden soll. Ist ein Bild vorhanden, so wird dieses in der Collapsing-Toolbar dargestellt \cite{collapsing}. Andernfalls wird kein Bild angezeigt.

\subsubsection{InputView}

\begin{figure}[H]
	\begin{center}
		\includegraphics[width=0.4\textwidth]{images/input.png}
		\caption{Beispiel einer View zum Anlegen eines Dozenten}
		\label{fig:input}
	\end{center}
\end{figure}

Für das neu Anlegen eines Dozenten oder auch zum bearbeiten muss entschieden werden, welche Attribute zum Anlegen benötigt werden. Auch hier ist es notwendig die Reihenfolge zu bestimmen. Jedoch kommen in dieser View für jedes Attribut noch die Möglichkeit hinzu ein Hint-Text anzugeben. Dieser Text beschreibt, was in der Android View EditText als Beschreibung für das bestimmte Attribut steht. Weiter sollte es die Möglichkeit geben, jedem Feld eine Nachricht mitzugeben, welche Angezeigt wird, wenn das Feld beispielsweise leer gelassen wird. Oder eine weitere Nachricht, wenn das Eingegebene nicht dem Erwarteten entspricht. Zum Beispiel wurde in das Feld für die E-Mail die Telefonnummer eingegeben. Oder es wurde ein regulärer Ausdruck mitgegeben und das Eingegebene entspricht nicht den Anforderungen, welche durch den regulären Ausdruck definiert wurden.

\subsubsection{Programmablauf und Klick-Aktionen}

Da das Enfield-Model bereits einen endlichen Automaten beschreibt, welcher den Programmablauf widerspiegelt, ist es nicht notwendig, diesen Ablauf noch einmal genauer zu definieren. Da der bereits definierte Ablauf übernommen wird.

Auch die Klick-Aktionen, das Geschehen, welches durch einen Klick auf ein bestimmtes Attribut ausgeführt werden soll, beschränkt sich auf Android Standard Aktionen. Beispielsweise das wechseln zu den Maps, zu einem E-Mail Client, dem Browser oder zum Anrufsmenü. Jede deser Aktion ergibt sich aus den Typen der Attribute, weswegen diese auch nicht weiter definiert werden müssen.

\subsection{Design der View-Meta-Modelle} \label{sec:resourceViews}

In den letzten Abschnitten der Arbeit wurde aufgezählt, was das Meta-Model alles Abdecken muss und das sowohl Android- als auch Polymer-seitig. In diesem Kapitel wird ein Meta-Model vorgestellt, welches die erwähnten Eigenschaften abdeckt.


\begin{figure}[H]
	\begin{center}
		\includegraphics[width=\textwidth]{images/metamodel.png}
		\caption{Aufbau der Views zur Erweiterung des Enfield-Models}
		\label{fig:meta-model}
	\end{center}
\end{figure}

Die Abbildung \ref{fig:meta-model} zeigt den Aufbau der Objekte, mit welchem das Enfield-Model erweitert wird. Die drei Views CardView, DetailView und InputView sind alles Instanzen von AbstraktResourceView. Jede der View, weiß welche Ressource sie darstellen soll. Dies passiert über die Zuordnung mit Hilfe des Ressourcennamens. Die drei Views, lassen sich in zwei Kategorien einteilen: Views, welche Informationen anzeigen und Views welche zur Eingabe von Informationen benötigt werden.
So gehören CardView und DetailView zur anzeigenden Views und die InputView zur zweiten Kategorie. 

\subsubsection{Anzeigende Views}
Diese View-Typen haben die Aufgabe in einer Liste alle Attribute zu halten, welche in der entsprechenden View angezeigt werden sollen. Dabei bestimmt die Reihenfolge, in welcher die Attribute in dieser Liste sind auch die Anordnung in der Oberfläche. Ist das erste Item in der Liste der Name, so wird dieser ganz oben in der View angezeigt.
Bei der DetailView jedoch gibt es nicht eine Liste mit den Attributen, sondern eine Liste mit Kategorien. Diese Kategorien, besitzen 
einen Namen und eine Liste mit den Attributen ihrer Kategorie. Die Darstellungsreihenfolge der Kategorien und deren Attribute ist analog zu der der CardView. Weiter besitzt die DetailView das Attribute \textit{image}, dieses Attribut wird hier aus der Liste der Attribute herausgezogen, da dieses Attribut bestimmt, ob die View eine Collapsing-Toolbar besitzen wird oder nicht. Wiederum haben beide Views das Attribut \textit{titleOfResource} dieses bestimmt welches Attribut unserer Ressource beispielsweise in der Toolbar angezeigt wird.

Auf die Polymer-spezifischen Attribute wird in dieser Arbeit nicht weiter eingegangen.

Mit Hilfe der Listen, Titelattributen und dem Bildattribut kann das Erscheinungsbild einer View schon ziemlich gut beschrieben werden. Jetzt bleibt fehlt noch die Möglichkeit, die Schriftgrößen, Schriftfarben, Klick-Aktionen und so weiter zu definieren.
Außerdem ist es bis jetzt nur möglich einfache Attribute anzuzeigen, eine horizontale Gruppierung ist noch nicht möglich. Um diese Anforderungen zu erfüllen, werden nicht Attribute in den Listen gespeichert sondern Ausprägungen von ResourceViewAttributen. 

Es gibt zwei Ausprägungen: ein SingleResourceViewAttribute und ein GroupedResourceViewAttribute.  Das SingleResourceViewAttribute ist für einfache Attribute, mit diesem ist es beispielsweise möglich den Titel eines Dozenten anzuzeigen. Das GroupedresourceViewAttribute ermöglicht die horizontale Gruppierung. Beide Objekte, bestimmen jedoch nicht die Design-spezifischen Eigenschaften des Attributs. Hierfür besitzen beide Attribut-Typen das Attribut DisplayViewAttribute.

Bei der SingleResourceViewAttribute ist diese Instanz von einem AbstractViewAttribute das einzige Attribut, beim GroupedResourceViewAttribute wiederum gibt es eine Liste von diesen DisplayViewAttributen, welche dann die anzuzeigenden Informationen widerspiegeln. Weitergehend besitzt diese Attribut-Art auch noch ein DisplayViewAttribute, welches die neu entstandene Gruppierung beschreiben soll.

Ein DisplayViewAttribute besitzt nun die Möglichkeit, die Schriftgröße und -farbe zu definieren. Die angegebene Farbe muss eine Farbe in hexadezimaler Darstellung sein, wird keine Farbe mitgegeben, wird die Default-Farbe der Anwendung genommen. In der Regel ist diese Schwarz.  Die Schriftgröße wiederum ist auf 3 Stufen beschränkt. Es gibt die Möglichkeit den Text in klein, normal und groß darzustellen. Per default ist normal eingestellt. Aus der Oberklasse AbstractViewAttribute besitzt das DisplayViewAttribute noch die Attribute \textit{attributeName}, dieses muss exakt so heißen wie in der Definition der Ressource beschrieben.
Mit dem \textit{attributeLabel} kann angegeben werden, wie dieses Attribut in der View angezeigt werden soll. Die Abbildung \ref{fig:detail} zeigt die Verwendung von den Labels, vor beispielsweise der E-Mailadresse des Dozenten steht \textit{E-Mail}, dieser String entspricht dem Label des Attributes. Muss angeben werden von welchem Typ das aktuell beschriebene Attribut ist.
Dies geschickt mit dem Attribut AttributeType. Es gibt folgende mögliche Typen: HOME, MAIL, LOCATION, PICTURE, PHONE\_NUMBER, TEXT, URL, DATE, SUBRESOURCE. Jeder Typ bestimmt die Eigenschaften des Attributes. Über diesen wird bestimmt welches Icon in der Karte vor dem entsprechenden Attribut angezeigt werden. Auch bestimmt er welche Aktion bei Klick ausgeführt werden soll. So wird bei einem Klick auf ein Attribut vom Typ LOCATION versucht die Anwendung Maps zu öffnen und den angezeigten Standort dort anzuzeigen. Ist das Attribute vom Typ SUBRESOURCE so wird für dieses Attribut ein Button angezeigt, dieser ermöglicht es dann zu der entsprechenden Subressource zu wechseln. Diese Klick-Aktionen müssen jedoch mit dem Attribut \textit{clickActionAndroid} erst aktiviert werden.

Manche Typen bringen noch ein paar andere Besonderheiten mit sich. So muss man beispielsweise bei einem URL-Attribut noch eine Beschreibung mitgeben, welche anstelle der Hyperlinks angezeigt werden soll. Bei einem Bild kann man beispielsweise noch bestimmen, ob dieses links oder rechts dargestellt werden soll. 

Das im Anhang befindliche Listing \ref{lst:detailview_impl} zeigt die Definition einer DetailView.

\subsubsection{Eingebende Views}

Bei der InputView gibt es wieder eine Liste, welche dieses mal InputViewAttribute mit der Oberklasse AbstractViewAttribute hält. Diese Liste bestimmt analog zu den anzeigenden Views die darzustellende Reihenfolge der Attribute. 

Neben dem \textit{attributeName} der wieder exakt dem Namen aus der Ressourcendefiniton entsprechen muss, besitzt das InputViewAttribute auch die Möglichkeit zu bestimmen, welcher Typ das aktuelle Attribut besitzt. Jedoch haben die Typen hier eine Andere Bedeutung als bei dem anderen View-Typ. So wird beispielsweise bei dem Type DATE kein EditText angezeigt, sondern der Nutzer hat die Möglichkeit das Datum über das DatePicker-Widget von Android einzugeben. 

Es ist jedoch für den Android-Client nicht möglich Bilder zu Ressourcen hinzuzufügen, oder diese zu Bearbeiten. Wird eine Subressource nicht in einer InputView der Oberressource bearbeitet oder neu angelegt. Dies ist dann in der entsprechenden View der Subressource möglich. Die anderen Typen beschränken das EditText-Widget auf die angegebenen Typen. So wird bei einem Klick auf ein PHONE\_NUMBER-Feld die Tastatur im Zahlenmodus ausgefahren und so weiter.

Einem InputViewAttribute muss zusätzlich ein \textit{hintText} mitgegeben werden, der im EditText des Attributs beschreibt, was in diesem Feld erwartet wird. Mit dem String \textit{missingText} kann dem Attribut mitgegeben werden, welche Nachricht dem Nutzer angezeigt wird, falls er versucht zu speichern ohne das entsprechende Feld auszufüllen. Mit der Kombination von \textit{checkPattern} und \textit{errorText} bekommt der Nutzer des Generators die Validierung des eingegebenen Attributes noch weiter zu verfeinern und auch dem Nutzer der Applikation ein Feedback zu geben, falls eine falsche Eingabe getätigt wurde.

Die Definition einer InputView wird im Anhang unter Listing \ref{lst:inputview_impl} dargestellt.

\subsection{Analyse und Design von allgemeinen Daten für eine Anwendung}

Dieses Kapitel behandelt die Informationen, welche eine Applikation neben den View-Beschreibungen zusätzlich benötigt, aber diese vom Kontext her nicht in einer der Views beschrieben werden können.

Ein Beispiel für eine solche Information wäre der \acf{url} für den Einstieg. Die Applikation benötigt diesen um zu wissen, unter welcher Adresse sich die anzuzeigenden Informationen zu finden sind. Ein weiteres Beispiel sind die Grundfarben der Applikation. Das Material Design gibt drei benötigte Grundfarben vor: \textit{colorPrimary}, \textit{colorprimaryDark} und \textit{colorAccent} diese Grundfarben wird um die Farbe für den Toolbar-Text erweitert.

Mit dem Wissen, konnte eine Erweiterung des Enfield-Models designed werden, welches in Abbildung \ref{fig:appspecifics} dargestellt ist.

\begin{figure}[H]
	\begin{center}
		\includegraphics[width=0.86\textwidth]{images/appspecifics.png}
		\caption{Aufbau des AppSpecifics Objekt  zur Erweiterung des Enfield-Models.}
		\label{fig:appspecifics}
	\end{center}
\end{figure}

 Über die Map \textit{additionalInformation} können zusätzlich weitere allgemeine Informationen an den Generator, zur Erzeugung der Anwendung, weitergegeben werden .

\section{Software-Generator}

Nachdem das Meta-Model nun klar ist, geht dieses Kapitel der Ausarbeitung auf die Funktionsweise des Generators ein. 
Es wird dargelegt wie der Generator aufgebaut ist und teilweise darauf eingegangen wieso dieser Weg der Generation gewählt wurde. Des weiteren wird das Java \acf{api} JavaPoet kurz vorgestellt \cite{poet}.

\subsection{JavaPoet}
JavaPoet ist ein Java \ac{api}, welches ermöglicht Java-Klassen zu generieren \cite{poet}. Hierfür wird die zu generierende Klasse programmiert. Mit Hilfe von nur ein paar Schlüsselwörtern ist es recht einfach möglich Klassen, Interfaces oder Methoden zu generieren. 

Da der größte Teil des Generators Java-Klassen erzeugen muss, ist dieses \ac{api} bestens für diesen Zweck geeignet. Sie erspart die aufwändige String-Manipulation. Durch die Nutzung wird auch bei der Ausführung des Programmes sichergestellt, das gültige Konventionen und Regeln eingehalten werden. So ist der grundsätzliche korrekte Aufbau einer Java-Klasse bereits sichergestellt.

Listing \ref{lst:poet} zeigt ein einfaches Beispiel zur Generierung einer Hello-World-Klasse und Listing \ref{lst:poet_result} zeigt das Ergebnis nach der Ausführung des Beispieles.

\begin{lstlisting}[label=lst:poet,
language=java,
firstnumber=1,
caption=Beispiel für die Generation einer Hallo-World-Klasse \cite{poet}.]				   
MethodSpec main = MethodSpec.methodBuilder("main")
	.addModifiers(Modifier.PUBLIC, Modifier.STATIC)
	.returns(void.class)
	.addParameter(String[].class, "args")
	.addStatement("$T.out.println($S)", System.class, "Hello, JavaPoet!")
	.build();

TypeSpec helloWorld = TypeSpec.classBuilder("HelloWorld")
	.addModifiers(Modifier.PUBLIC, Modifier.FINAL)
	.addMethod(main)
	.build();

JavaFile javaFile = JavaFile.builder("com.example.helloworld", helloWorld)
	.build();
\end{lstlisting}

\begin{lstlisting}[label=lst:poet_result,
language=java,
firstnumber=1,
caption=Ergebnis der Generation von Listing \ref{lst:poet} \cite{poet}.]				   
package com.example.helloworld;

public final class HelloWorld {
	public static void main(String[] args) {
		System.out.println("Hello, JavaPoet!");
	}
}
\end{lstlisting}

\subsection{Generierung anderer Daten-Typen}

Neben Java-Klassen besitzt der Sourcecode einer Android Applikation auch XML-Dateien und Gradle-Dateien. Für diese Typen muss eine andere Möglichkeit der Generierung gewählt werden. Hierfür liefert \acf{gemara} mit der Klasse GeneratedFile eine Möglichkeit. Diese Klasse liefert die die beiden Methoden \textit{append(String contet)} und \textit{appendln(String content)}. Welche es ermöglichen jedes beliebige textbasiertes File-Format zu generieren. Ein GeneratedFile Objekt erzeugt eine Datei, welcher mit den beiden erwähnten Methoden Strings hinzugefügt werden können, dies ermöglicht es jede beliebige Textstruktur zu erzeugen. Jedoch liefert diese Klasse keinerlei Validierung, die Datei wird generiert egal ob die Struktur gültig ist oder nicht.

So würde Listing \ref{lst:append} eine Datei \textit{test.xml} im Verzeichnis \textit{generated} erzeugen. Diese erzeuge Datei wird in Listing \ref{lst:append_result} dargestellt.

\begin{lstlisting}[label=lst:append,
language=java,
firstnumber=1,
caption=Beispiel eine GeneratedFile-Instanz zur Erzeugung einer XML-Datei.]				   
public class FileGenerator extends GeneratedFile {

	@Override
	public void generate() {
		appendln("<?xml version=\"1.0\" encoding=\"utf-8\"?>");
		appendln("<menu xmlns:android=\"http://schemas.android.com/apk/res/android\" xmlns:app=\"http://schemas.android.com/apk/res-auto\">");
		appendln("<item android:id=\"@+id/saveItem\"");
		appendln("android:title=\"@string/save\"");
		appendln("app:showAsAction=\"always\"\\>");
		appendln("<\\menu>");
	}

	@Override
	protected String getFileName() {
		return "test.xml";
	}

	@Override
	protected String getDirectoryName() {
		return "/generated";
	}
}
\end{lstlisting}

\begin{lstlisting}[label=lst:append_result,
language=xml,
firstnumber=1,
caption=Erzeugte XML-Datei durch den Quellcode von Listing \ref{lst:append}.]				   
<?xml version="1.0" encoding="utf-8"?>
	<menu xmlns:android="http://schemas.android.com/apk/res/android"
		xmlns:app="http://schemas.android.com/apk/res-auto">
		<item android:id="@+id/saveItem"
			android:title="@string/save"
			android:icon="@drawable/ic_done"
			app:showAsAction="always"/>
	</menu>
\end{lstlisting}

\subsection{Aufbau der zu generierenden Applikation}

Um den Generator möglichst zu vereinfachen, ist es hilfreich, eine Referenzimplementierung der gewünschten Applikation mit all ihren Funktionen und Anforderungen zu entwickeln. Bei einer anschließenden Quellcode-Analyse sollte darauf geachtet werden, die einzelnen Klassen soweit zu abstrahieren, das eine Einteilung in generischen und spezifischen Quellcode erfolgen kann. Der generische Quellcode ist einfacher zu generieren, da dieser statisch ist und sich für alle folgenden Implementierungen nicht verändert. Es können auch Überlegungen angestrebt werden, diese generischen Klassen einfach im Generator abzulegen und bei Bedarf zu kopieren. Diese Methode wurde verworfen, da sich andernfalls jedes mal die kopierten Klassen via String-Manipulation bearbeitet werden müssten. Die minimale Änderung welche jedes mal getroffen werden müsste, wäre das Anpassen der Package Anweisung am Anfang der Java-Klassen und die Anpassung der Import-Anweisungen. Eine weitere Überlegung wäre es, diese Klassen in eine Android Bibliothek auszulagern, und diese dann in jede Anwendung zu importieren. Auch von dieser Möglichkeit wurde in der ersten Version abgesehen, da die Applikation bereits aus zwei Komponenten besteht. Der Applikation an sich und einer Bibliothek, welche die Android-Komponenten für die Anwendung enthält. Um die Komplexität zu reduzieren werden die benötigten generischen Klassen als Teil der eingebunden Bibliothek jedes mal aufs neue generiert.

Die Referenzimplementierung für diese Arbeit beinhaltet folgende Features:
\begin{center}
	\begin{tabular}{p{0.4\textwidth}|p{0.4\textwidth}}
		\textbf{Ressource: Dozent} & \textbf{Ressource: Amt} \\ \hline
		\begin{itemize}
			\item Anzeige einer Liste mit Dozenten
			\item Anlegen neuer Dozenten
			\item Anzeigen eines Dozenten in Detail
			\item Bearbeiten eines Dozenten
			\item Löschen eines Dozenten
		\end{itemize}
		&
		\begin{itemize}
			\item Anzeigen einer Liste von Ämtern eines Dozenten
			\item Anlegen neuer Ämter für einen Dozent
			\item Anzeigen eines Amtes eines Dozenten
			\item Bearbeiten eines Amtes eines Dozenten
			\item Löschen eines Amtes eines Dozenten
		\end{itemize}
		\\
	\end{tabular}
\end{center}

Der Aufbau der Referenzimplementierung wird in Abbildung \ref{fig:lecturer} dargestellt. Das Schaubild verdeutlicht das Verhältnis von generischen (weiße Kästen) und spezifischen (rote Kästen) Klassen. Die Anzahl der gleichbleibenden Klassen ist mit etwa 60 Prozent bereits höher als der Anteil an spezifischen Klassen. Je höher der Anteil dieser unveränderlichen Klassen, desto geringer wird die Komplexität des Generators. Da der Aufwand eine spezifische Klasse zu erzeugen mehr Logik benötigt, als eine Klasse, welche immer gleich bleibt.

Daneben zeigt die Abbildung \ref{fig:lecturer} aus dem Anhang, auch noch die Aufteilung der Klassen in Klassen der Applikation (gestrichelte Kästen) und Klassen der Bibliothek (solide Kästen). Die Applikation an sich besteht nur aus ein paar wenigen Fragmenten und Aktivities, welche alle projektspezifisch sind. Der komplette generische Quellcode befindet sich in der Bibliothek. Des weiteren befinden sich dort auch die spezifischen Komponenten, beispielsweise der \textit{LecturerInputView}. Diese Komponente, kann in den Fragmenten zur Bearbeitung oder Neuanlage eines Dozenten dann mit wenigen Zeilen Programmcode verwendet werden.

Diese Art der Aufteilung ermöglicht es das ein Applikation Entwickler sich die Komponente, für das Anzeigen, Bearbeiten, Löschen und der Neuanlage generieren lassen kann. Diese Komponenten jedoch beliebig in seiner eigenen Applikation verwenden kann.

\subsection{Aufbau des Generators}

\begin{figure}[H]
	\begin{center}
		\includegraphics[width=\textwidth]{images/Welling.png}
		\caption{Aufbau des Android-Generators Welling}
		\label{fig:welling}
	\end{center}
\end{figure}


Die Klasse ApplicationGenerator, ist der Einstiegspunkt des Projekts. Sie erwartet im Konstruktor ein Enfield-Model Objekt. Wie der Abbildung \ref{fig:welling} entnommen werden kann, so lässt sich das Projekt in drei Teilbereiche gliedern. Der erste Bereich erzeugt ein AppDescription Objekt (Abbildung \ref{fig:appDescription}) der zweite Bereich befasst sich mit allgemeinen Vorbereitungen, die getroffen werden müssen. Der Letzte iteriert über die States, und generiert nach bedarf die benötigten Klassen.

Die ApplicationGenerator Klasse verfügt über eine öffentliche Methode \textit{generate}. Beim Aufrufen dieser Methode, werden die einzelnen Generatoren, für den allgemeinen Bereich angestoßen. Weiterhin wird das iterieren über die States des Enfield-Model begonnen. Zum Schluss wird noch das AppDescription Objekt ausgewertet, und die darin enthaltenen Informationen in Dateien geschrieben und an die entsprechende Stelle im Projekt gespeichert.

\subsubsection{Erstellung der AppDescription}

\begin{figure}[H]
	\begin{center}
		\includegraphics[width=\textwidth]{images/AppDescription.png}
		\caption{Aufbau des AppDescription Objekts.}
		\label{fig:appDescription}
	\end{center}
\end{figure}

In der AppDescription werden alle Daten durch den Generator gereicht, welche an vielen Stellen benötigt werden. 
An vielen Stellen wird beispielsweise dir Name der Anwendung oder der Bibliothek benötigt. In jeder Java-Klasse wird der Paket Name benötigt, da dieser in der Bibliothek und in der normalen Applikation verschieden sind müssen diese für beide mitgeführt werden. Auch muss der Generator wissen, unter welchen Verzeichnissen die aktuelle Datei egal ob Java Klasse oder XML-Datei gespeichert werden soll. 
Diese Dateien können einfach aus dem Enfield-Model abgelesen werden. Auch die Ressourcen und die jeweiligen Subressourcen können direkt aus dem Meta-Model entnommen werden.  Dies ist der Teil des initialisieren der AppDescription. Alle bereits jetzt verfügbaren Informationen werden der AppDescription zugewiesen. 

Neben diesen Daten, die an mehreren Stellen bei der Generierung benötigt werden, gibt es Dateien in einer Android-Applikation, die sich mit dem Generieren aufbauen. Ein Beispiel für eine solche Datei ist die \textit{strings.xml}. 
Es wird in dem generierten Projekt zwei davon geben. Eine im Bereich der Applikation selbst und eine weitere im Bereich der Bibliothek. Diese Dateien enthalten neben dem Applikationsnamen beziehungsweise des Bibliotheksnamen auch viele Strings, die erst beispielsweise in einem Fragment auftauchen. Jedoch müssen die benötigten Datensätze in der \textit{strings.xml} eingetragen werden. Anstelle dies jedes mal wenn im Ablauf des Generierens ein String auftaucht, eine bereits generierte Datei zu erweitern, wird der Datensatz in der AppDescription unter dem AppString \textit{appString} beziehungsweise dem AppString \textit{libString} hinterlegt.

Auch das AndroidManifest wächst mit der Anwendung. So muss jede benutzte Aktivity dort eingetragen sein. Andernfalls kann diese nicht genutzt werden. Am Anfang des Generierens ist die genaue Anzahl und die genauen Namen der Aktivities unbekannt, weswegen der Generator diese beim Erzeugen zur AppDescription hinzufügen muss. 

Das Attribut \textit{appDeclareStyleable} enthält alle Custom Attribute, welche wie, im Kapitel \ref{sec:custom_view}, in die \textit{attr.xml} eingetragen werden müssen.

Da die Anwendung, welche generiert wird auch den \acf{rest} Ansätzen entsprechen soll, muss diese wissen welche Relationstpyen zu welchen Endpunkten gehören. Anfangs sind diese jedoch auch unbekannt und werden erst im weiteren Verlauf beim iterieren über die States bekannt und zur AppDescription hinzugefügt.

So wächst die AppDescription über den gesamten Prozess des Generierens. Ganz am Ende, werden die gesammelten Daten in die entsprechenden Dateien an den jeweiligen Orten gespeichert. Das Verwenden und weiterreichen eines AppDescription Objekts reduziert die Komplexität des Generators. Dieser muss nicht bei jeder Ergänzung einer der beschriebenen Dateien diese Aufrufen, den neuen Datensatz aufwändig hinzufügen und die Datei wieder abspeichern. Sondern der Generator muss nun die Datei nur einmal schreiben, da er jetzt alle von der Android Anwendung benötigten Informationen besitzt.

\subsubsection{Vorbereitung und Generierung allgemeiner Dateien}

Der Bereich zur Vorbereitung und Generierung der allgemeinen Dateien gliedert sich ebenfalls in 3 Bereiche. Der erste Bereich kümmert sich um alle Dateien die von Gradle benötigt werden. 

Er kopiert Daten wie die \textit{gradlew.bat}, \textit{gradlew}, \textit{build.gradle} und den Gradle Wrapper. 
Neben dem Kopieren, werden sowohl für die Applikation, Bibliothek als auch für das Gesamtprojekt die spezifischen Dateien generiert. So wird beispielsweise auf der Projektebene eine \textit{settings.gradle} erzeugt oder in der Applikation sowie in der Bibliothek jeweils eine \textit{build.gradle}.

In der Sektion der Vorbereitung für die Applikation an sich, werden Dateien erzeugt, die jede Applikation benötigt unabhängig von ihrem Aufbau oder den Features. Es wird beispielsweise die MainActivity erzeugt, oder die XML-Dateien, welche für die Transitionsanimationen verantwortlich sind. Auch die \textit{styles.xml} wird erzeugt. Am Schluss werden noch die \textit{mipmap}-Ordner kopiert und an die richtige Stelle verschoben.

Der Bereich, welcher die Bibliothek initialisiert, ist der Größte. Er generiert alle generell benötigten Klassen. Darunter fallen die Klassen für die Netzwerkkommunikation, Die Klasse für das Link-Objekt sowie das Interface \textit{Resource}. Es werden des  werden auch die größten Teile der in der Abbildung \ref{fig:lecturer} abgebildeten generischen Klassen erzeugt. Auch werden die grundsätzlichen CustomViews bereits erzeugt. Dazu gehören auch noch die benötigten XML-Dateien. So kann für die Bibliothek beispielsweise das Manifest bereits erzeugt werde, da hier keine Activities registriert werden müssen. Nach dem Ausführen des \textit{PrepareLibGenerators} steht, das Grundgerüst der Bibliothek. Diese enthält nun alle bereits vorab erzeugbaren und benötigten Dateien, welche unabhängig von der gewünschten Funktion der Applikation benötigt werden. 

Dieser gesamte Teilbereich des Projekts befasst sich damit ein Grundgerüst für die komplette Android Applikation zu erzeugen und vorab bereits alle benötigten Dateien bereit zu stellen. Die generierten Klassen haben jedoch noch keinerlei Programmlogik, die den spezifischen Ablauf der zu generierenden Anwendung steuert.

\subsubsection{Iterieren über die States}

Der Teilbereich, der sich mit dem iterieren über die einzelnen States beschäftigt ist der komplexeste Bereich des Generators. Er ist dafür verantwortlich, das zu jedem State die alle benötigten Klassen und Dateien generiert werden. 

Um diese Anforderung zu erfüllen, nutzt er den Visitor \textit{IStateVisitor}, welcher durch das Enfield-Model zur Verfügung gestellt wird. Außerdem wird auch der Visitor \textit{VisitStatesOnlyOnce} benutzt. Dieser zweite Visitor stellt sicher, das jeder State nur einmalig Besucht wird. Würde der Generator einfach nur über die Transitionen der States gehen, könnte es passieren, das er in eine Endlosschleife gerät.

Gelangt der Generator zu einem State, wird mit dem \textit{ISateVisitor} identifiziert, von welchem Typ dieser ist. Ist es ein State, welcher einen GET-Request auf eine einzelne Ressource oder auf eine Collection beschreibt, oder beschreibt er einen POST-, PUT- oder DELETE-Request.  Nach dieser Identifikation, wird bei jedem State, außer dem DELETE-State, eine Klasse für die in diesem State betroffene Ressource erzeugt. Hierfür wird der \textit{ResourceGenerator} benutzt. Auch wenn dabei die Ressource mehrfach angelegt werden würde. Der Generator überschreibt eine bereits angelegte Ressource einfach. Diese Redundanz garantiert das auf jeden Fall eine Ressource zum betreffenden State existiert. 

Neben diesen Ressource-Klassen, wird auch ein StateHolder-Objekt erstellt. Die Abbildung \ref{fig:stateHolder} repräsentiert dieses. 

\begin{figure}[H]
	\begin{center}
		\includegraphics[width=0.3\textwidth]{images/StateHolder.png}
		\caption{Aufbau des StateHolder Objekts.}
		\label{fig:stateHolder}
	\end{center}
\end{figure}

Dieses Objekt wird für jeden einzelnen State angelegt, es enthält alle States, welche über die Transitionen erreicht werden können.
So weiß der Generator genau, ob beispielsweise ein Button angezeigt werden muss, der eine Neuanlage einer Ressource ermöglicht. Diese Informationen stecken zwar auch im Enfield-Model, jedoch müsste jedes mal wenn überprüft werden soll welche Folgestates ein State besitzt, über alle States iteriert werden. Das StateHolder-Objekt beschreibt sozusagen eine Landkarte für jeden einzelnen State.

Der State, welcher für das Löschen einer Ressource verantwortlich ist, ist der einfachste zum generieren. Hierfür muss nur ein DialogFragment erzeugt werden, der für das Löschen verwendet wird. 

Für die anderen States, werden mehr Klassen und Dateien benötigt. Außerdem werden die ResourceViews (Kapitel \ref{sec:resourceViews}) benötigt, die jedem State angehängt sind. Zur Identifizierung der einzelnen ResourceViews wird wiederum mit dem Visitor-Pattern gearbeitet. Die Klasse der ResourceView stellt den Visitor \enquote{ResoruceViewVisitor} zur Verfügung. 
Nachdem bekannt ist welche der drei ResourceView-Typen im entsprechenden State verwendet wurde, kann einer der Komponentengeneratoren: InputViewGenerator, CardViewGenerator oder DetailViewGenerator alle notwendigen Dateien generieren.

\begin{figure}[H]
	\begin{center}
		\includegraphics[width=\textwidth]{images/Lecturer-Swimlines.png}
		\caption{Aufbau der Dozenten Applikation mit Einteilung in spezifische States.}
		\label{fig:swimlines}
	\end{center}
\end{figure}

In Abbildung \ref{fig:swimlines} ist die Applikation für Dozenten noch einmal abgebildet. Zur Vereinfachung wurde bei diesem Diagramm jedoch die Ressource Ämter mit ihren zugehörigen Klassen weggelassen.

Der Bereich \enquote{Update} und der Bereich \enquote{Create} werden hierbei vom InputViewGenerator, der Bereich \enquote{Read Collection} von CardViewGenerator und der Bereich \enquote{Read Single} vom DetailViewGenerator erzeugt.

Jeder der einzelnen Generatoren ist ein Zusammenschluss von vielen Teilgeneratoren. Es werden dabei in einem der Generatoren nicht nur die Java-Klassen für die Applikation oder die Bibliothek, sondern auch alle benötigten XML-Dateien erzeugt.

So ist beispielsweise der DetailViewGenerator dafür verantwortlich, dass auf der Seite der Applikation, die \enquote{LecturerDetailActivity} inklusive ihrer XML-Datei erzeugt wird. Er muss weitergehend auch diese Aktivity in die AppDescription im Bereich des Manifestes hinterlegen. Im Bereich der Bibliothek muss dafür gesorgt werden, dass die generischen Klassen \enquote{ResourceDetailActivity}, \enquote{ResourceDetailView} sowie die spezifischen Klassen: \enquote{LectuererDetailView}, \enquote{LectuererDetailAdapter}, \enquote{LectuererDetailViewHolder}, \enquote{LectuererDetailCardView} erzeugt werden. Zu all diesen Klassen müssen mögliche Strings oder CustomViews in die AppDescription aufgenommen werden. Wiederum müssen auch die entsprechenden XML-Dateien erzeugt werden. 

Jeder Generator besitzt mehrere Möglichkeiten, welche Klassen generiert werden müssen. So entscheidet beispielsweise ob die Ressource ein Bild besitzt oder nicht über den Verhalt, ob eine Activity mit einer CollapsingToolbar verwendet wird oder ob ein einfaches Fragment zur Detailanzeige ausreichend ist.

Selbst die Generatoren auf der untersten Ebene, welche die einzelne Klassen erzeugen, wissen mit Hilfe von dem mitgegebenen StateHolder, ob beispielsweise Menüeinträge für das Löschen oder das Bearbeiten von Ressourcen benötigt werden. Diese Generatoren richten sich auch nach den übergebenen RessourceViews. Auf dieser Ebene haben die vom Benutzer des Generators mitgegebenen Informationen zum Aussehen, Einfluss. Hier werden die benötigten Attribute der Ressource hinzugefügt, und deren Aussehen in den entsprechenden XML-Dateien beschrieben.


\section{Bauen und ausführen der generierten Android Applikation}

Wurden alle benötigten Dateien der Applikation erzeugt, gibt es zwei Möglichkeiten, die Applikation zu bauen und anschließend auf einem Android-Endgerät zu installieren.

Variante 1: Importieren der generierten Dateien in eine \acf{ide} beispielsweise in Android Studio. Dort wie bereits bekannt, die Anwendung bauen und auf einem sich im Entwicklermodus befindlichen Android-Endgerät installieren.

Variante 2: Die Applikation mit Hilfe des Makefile bauen und installieren. Hierfür muss ebenfalls ein Android-Endgerät im Entwicklermodus an dem entsprechenden Computer angeschlossen sein.

\newpage

\begin{lstlisting}[label=lst:make,
language=xml,
firstnumber=1,
caption=Makefile für das Bauen und Installieren der erzeugten Applikation.]				   
APK = gemara/android/src-gen/generated/app/build/outputs/apk/app-debug.apk

all: debug install

debug:
cd gemara/android/src-gen/generated && chmod 777 gradlew && ./gradlew clean assembleDebug

install:
adb $(TARGET) install -rk $(APK)
\end{lstlisting}

Listing \ref{lst:make} zeigt das Makefile, dieses bietet die Möglichkeit entweder mit dem Befehl \textit{make} eine Debug-Version der Anwendung zu bauen und zu Installieren, oder mit dem Befehl \textit{make debug} ausschließlich die Applikation zu bauen beziehungsweise mit dem Befehl \textit{make install} die bereits gebaute Applikation zu installieren.


\chapter{Evaluierung anhand einer Beispielanwendung}
Dieses Kapitel befasst sich mit Pro und Contra des Generierens einer Android Applikation, nach dem in dieser Arbeit vorgestellten Methode.
Hierfür wird die Erstellung und die Benutzung des \textit{Meta-Modells} genauer beschrieben, dabei werden die Vorteile und Nachteile des Modells dargestellt. Anschließend wird die Komplexität der Anwendung genauer betrachtet und die Zeitaufwände für die Entwicklung und Wartung des Generators erörtert.

\section{Erstellung und Nutzung des Meta-Modells}

Im Vergleich zum Umfang der Entwicklung einer kompletten Anwendung, reduziert die Nutzung des Generators den Aufwand erheblich. Die im Anhang befindlichen Listings \ref{lst:detailview_impl}, \ref{lst:inputview_impl} und \ref{lst:cardview_impl} zeigen den kompletten Aufwand der Beschreibung der Anwendung. Auch wenn diese nur ein Teil der benötigten Informationen sind, kann der Rest vernachlässigt werden. Der zusätzlich benötigte Teil ist die Kernbeschreibung der Generierung des \textit{Backends}, so ist dieser semantisch stark diesem Bereich zugeordnet und dort essentiell. Durch diesen Umstand werden diese Informationen als gegeben betrachtet.

Die Fehleranfälligkeit bei der Nutzung des \textit{Meta-Modells} im Gegensatz zur Entwicklung einer kompletten Anwendung ist wesentlich geringer. Die Erzeugung des \textit{Meta-Modells} ist sehr viel eingeschränkter in seinen Möglichkeiten, dadurch wird die Möglichkeit, Fehler zu machen bedeutend reduziert. Das \textit{Meta-Modell} liefert in gewisser weise einen Plan, wie etwas beschrieben werden muss. Bei der Eigenentwicklung einer Anwendung ist der Entwickler viel freier in der gesamten Handhabe, was das Fehlerpotenzial erhöht.

Jedoch bringt diese Einschränkung durchaus auch Nachteile mit sich. So ist es im Moment beispielsweise nur möglich einer Ressource ein oder kein Bild zuzuweisen. Auch kann der Nutzer lediglich bestimmen, ob dieses Bild in der Karte einer Ressource, in der Liste mit allen Ressourcen dieser Art, auf der linken oder rechten Seite angezeigt werden soll. Für die Detail-Ansicht hat der Nutzer des Generators keine Möglichkeit zu bestimmen wie das Bild angezeigt werden soll.
Auch bleibt zur Anzeige der Informationen einer Ressource lediglich die Möglichkeit diese in Listenform darzustellen. Sprich er kann nur die Reihenfolge und eine Mögliche Gruppierung bestimmen und in der Detail-Ansicht müssen diese Informationen zusätzlich in Kategorien gruppiert sein. 

In der aktuellen Version kann der Benutzer des Generators keine eigene Funktionen mit dem Klick auf ein Attribut ausführen, sondern ausschließlich ein Subset von vordefinierten Funktionen. Das gleiche gilt auch für die Icons, welche in der Karte vor den einzelnen Attributen sichtbar sind. Es gibt im Moment keine Möglichkeit dort eigene Icons anzeigen zu lassen.

\section{Zeitaufwände und Komplexität}

Der gesamte Generator ist in seiner Entwicklung sehr zeitaufwändig. Durch das Analysieren der Anforderungen und des entwickeln einer \textit{\acf{dsl}}, ist dieser Zeitaufwand nur dann gerechtfertigt, wenn mithilfe des Generators viele Anwendungen generiert werden können. Die Neuentwicklung einer Android-Applikation mit dem oben beschriebenen Funktionsumfang bedarf einen ungefähren Zeitaufwand von ca. drei Arbeitstagen. Wobei der Zeitaufwand für den Generator ca. zwei bis zweieinhalb Monate beträgt.

Der Aufwand für die Wartung des Generators ist auch ziemlich hoch. Da Android sich ständig weiterentwickelt und eine Umstellung auf das \textit{OpenJDK} erfolgen soll \cite{jdk}, ist es anzunehmen, das sich in Zukunft auch die Art und Weiße der Android Programmierung ändern wird. Sollte dies der Fall sein, dann müssten im kompletten Generator Anpassungen gemacht werden. Diese sind sehr zeitaufwändig, da der Generator, wie Abbildung \ref{fig:welling} verdeutlicht, sehr komplex ist.

Auch ist die Komplexität der erzeugten Applikation sehr hoch. Um die Komplexität des Generators zu reduzieren, wurde das mit der Komplexität der Applikation bezahlt. Diese Komplexität rührt daher, dass zur Einteilung in spezifische und generische Codebereiche, der vorhandene Programmcode so weit wie möglich abstrahiert wurde. Diese Abstraktion führt dazu das die Anzahl der benötigten Klassen mindestens verdoppelt, da man davon ausgehen kann, das zu jeder spezifischen Klasse mindestens eine generische Klasse erzeugt werden muss. Es können zwar einige abstrakten Klassen von mehreren spezifischen Klassen benutzt werden, jedoch ist in diesem Beispiel diese Wiederverwendung vernachlässigbar. Da die Anzahl der mehrfach benutzen abstrakten Klassen gegenüber dem direkten Vergleich von spezifischen zu generischen Klassen kaum ins Gewicht fällt. Mit der Anzahl der Klassen, haben sich auch die Abhängigkeiten innerhalb der Klassen erhöht. Dadurch ist beispielsweise die Fehleranalyse vor allem während der Entwicklung sehr aufwändig. Auch das Vorgehen, dass nicht eine Anwendung im üblichen Sinn erzeugt wird, sondern das Komponenten in einer Bibliothek erzeugt werden, steigert den Umfang der Applikation. Bei hardware-schwächeren Endgeräten, könnte dieser Umstand zu Problemen mit der Performance führen. Diese Performanceprobleme entstehen durch die größere Verschachtlung einzelner \textit{View}-Klassen. Die \textit{View}-Klasse wir oft ohne das Bewusstsein, um deren Komplexität verwendet. Diese Klasse hat die Aufgabe den anzuzeigenden Inhalt soweit aufzubereiten um ihn auf dem Display anzuzeigen. Verschachtelt man diese Klasse, wird der Rechenaufwand für das Endgerät bedeutend erhöht. Aus diesem Grund sind flache \textit{View}-Strukturen vorzuziehen. Deshalb benötigt das Endgerät mehr Rechenleistung um die Anwendung ohne Ruckler darzustellen.

Gegen den großen zeitlichen Aufwand spricht die Einsparung von Zeit beim Generieren neuer Applikationen. Die Beschreibung eines \textit{Meta-Modells} mit allen benötigten Angaben und Informationen, welches die Erzeugung einer Backend-Anwendung inkludiert, benötigt nur noch ungefähr eine Stunde. 

\chapter{Zusammenfassung}
Im diesem Kapitel wird die gesamte Ausarbeitung noch einmal zusammengefasst, dabei spiegelt diese Zusammenfassung auch den noch einmal den Aufbau der Arbeit dar. Abschließend werden mögliche Ausblicke vorgestellt. Diese beinhalten Erweiterungen, um den der Software-Generator erweitert werden könnte.

\section{Zusammenfassung}
Im Rahmen dieser Arbeit wurde ein Generator für Android Applikationen als Teilprojekt des Generators \acf{gemara} entwickelt.
Ein Ziel dabei war dem Leser die grundsätzliche Problematik bei der Entwicklung von Software-Generatoren näher zu bringen.
Es wurde erklärt weswegen ein Generator ein \textit{Meta-Modell} benötigt, und mögliche Modelle vorgestellt. 

Bei der Vorstellung der \textit{Meta-Modelle} wurde aufgezeigt, welche Vorteile und Nachteile das jeweilige Modell besitzt. So wurde bei dem Android-spezifischen Modell gezeigt, dass dieses flexibler im Bereich der Funktionalitäten und des Ablaufes innerhalb der Anwendung ist. Jedoch ist es nicht oder nur schwer möglich dieses Modell für einen anderen \textit{Client} mitzuverwenden. Bei der Vorstellung des universellen Modells wurden die bei der Analyse angewendeten Fragen aufgezeigt. Die Verdeutlichen sollen, was bei der Entwicklung von \textit{Meta-Modellen} alles berücksichtigt werden muss.

Es wurde darauf eingegangen, dass das gegebene \textit{Enfield-Meta-Modell} nicht an einer Stelle, sondern an den entsprechenden Stellen in den \textit{States} erweitert wird. Dies hat zum Vorteil das der Generator durch das iterieren über die \textit{States} mit Hilfe der Transitionen, eine Art Fahrplan der Applikation besitzt und zur passenden Stelle alle relevanten Informationen zur Verfügung hat.

Anhand von Codebeispielen wurde gezeigt, wie die \textit{Views} modelliert werden müssen und wie das Ergebnis aussieht. Besonders wurde darauf eingegangen, welche Möglichkeiten der Benutzer des Generators besitzt, um die \textit{Views} zu gestalten. Zu den Gestaltungsmöglichkeiten der Oberfläche wurde außerdem aufgezeigt welche möglichen Interaktionen bei einem Klick ausgelöst werden können.

Auch wurde dem Leser näher gebracht, wie der Generator für eine Android Applikation funktioniert. Es wurde das Java \textit{\acf{api}} \textit{JavaPoet} vorgestellt, mit wessen Hilfe die Java-Klassen erzeugt werden können. Daneben wurde auch aufgezeigt wie die anderen Dateien erzeugt werden können. 
Neben dem reinen Erzeugen wurde der Ablauf im Generator vorgestellt. Es wurde gezeigt das sich dieser in drei Bereiche gliedert.
Jeder dieser Bereiche wurde vorgestellt und auf seine Besonderheiten hingewiesen. Dadurch sollte ein Verständnis über die Funktionsweise vermittelt werden.  

\section{Ausblick}

Im letzten Kapitel der Ausarbeitung sollen Ideen und mögliche Erweiterungen es Meta-Modells sowie des Software-Generators für Android Applikationen vorgestellt werden.

In der ersten Version des Generators ist es bisher nur möglich eine Ressource als \textit{Primärressource} zu definieren. Jedoch würde es die Möglichkeit geben, mehrere Ressourcen zu definieren und in der Applikation mit Hilfe eines \textit{Navigation-Drawers} zwischen diesen umzuschalten. Der Grundstein dafür ist bereits in dieser Version gelegt worden. Neben den drei, in dieser Arbeit beschriebenen, \textit{ResourceViews} wurde bereits eine vierte \textit{View} im \textit{Meta-Modell} eingefügt. Die \textit{NavigationDrawerRessourceView} mit deren Hilfe der \textit{Drawer} in der Applikation beschrieben werden könnte.

Außerdem sind im Moment die \textit{Views} auf ein Bild beschränkt, in einer späteren Version,  könnte diese Begrenzung  aufgehoben werden und dadurch einer Ressource mehrere Bilder als Attribute zugeteilt werden. Dafür müsste jedoch auch das Modell dahingehend erweitert werden, das der Generator weiß, welches Bild als Titelbild verwendet wird. Dieses würde dann weiterhin in der \textit{CollapsingToolbar} der Detailansicht angezeigt werden. Da die \textit{CollapsingToolbar} ein Style-Element von Material Design ist, sollte dieses so beibehalten werden. Jedoch müsste überlegt werden wie die zusätzlichen Bilder angezeigt werden sollen.

Auch wurde in der Ausarbeitung darauf eingegangen, das einem Attribut in einer \textit{InputResourceView} ein \textit{checkPattern} sowie ein \textit{errorText} mitgegeben werden kann. Jedoch werden aktuell diese Eigenschaften nicht zur Validierung der Eingabe herangezogen. Zusätzlich könnten für die Checks noch angegeben werden ob es optionale Eingabefelder gibt. Im Moment müssen alle angegebenen Felder befüllt werden.

\newpage

Da es für Android-Anwendungen eher unüblich ist Bilder zu einer Ressource, durch das Hochladen dieses, hinzuzufügen, wurde im ersten Entwurf auf das Feature verzichtet. In Zukunft wäre es jedoch denkbar, diese Möglichkeit zu unterstützten. 
Ein weitere nützliche Erweiterung wäre die Suche nach einer bestimmten Ressource. Dieses Feature war zwar Anfangs bereits angedacht, wurde jedoch erst einmal wegen einer geringeren Priorität hinten angestellt.


\backmatter
%%%%%%%%%%%%%%%%%%%
%% create figure list
%%%%%%%%%%%%%%%%%%%

\listoffigures
\addcontentsline{toc}{chapter}{Verzeichnisse}			

%%%%%%%%%%%%%%%%%%%
%% create tables list
%%%%%%%%%%%%%%%%%%%
%\listoftables

%%%%%%%%%%%%%%%%%%%
%% create listings list
%%%%%%%%%%%%%%%%%%%
%\lstlistoflistings
%\addcontentsline{toc}{chapter}{Listings}				

\printbibliography
\addcontentsline{toc}{chapter}{Literatur}
%%%%%%%%%%%%%%%%%%%
%% declaration on oath
%%%%%%%%%%%%%%%%%%%

\addchap{Eidesstattliche Erklärung}

Hiermit versichere ich, dass ich die vorgelegte Bachelorarbeit selbstständig verfasst und noch nicht anderweitig zu Prüfungszwecken vorgelegt habe. Alle benutzten Quellen und Hilfsmittel sind angegeben, wörtliche und sinngemäße Zitate wurden als solche gekennzeichnet.

\vspace{20pt}
\begin{flushright}
$\overline{~~~~~~~~~~~~~~~~~\mbox{\BaAuthor, am \today}~~~~~~~~~~~~~~~~~}$
\end{flushright}

\appendix

\chapter{Anhang}

\newpage
\begin{figure}[H]
	\begin{center}
		\includegraphics[width=\textwidth, angle=90]{images/Lecturer.png}
		\caption*{Aufbau der Referenzimplementierung.}
		\label{fig:lecturer}
	\end{center}
\end{figure}

\newpage

\begin{lstlisting}[label=lst:detailview_impl,
language=java,
firstnumber=1,
caption=Erstellung einer DetailView.]				   
...
DetailView detailView;

	try {
		List<Category> categories = new ArrayList<>();

		DisplayViewAttribute nameAttribute = new DisplayViewAttribute("name", ViewAttribute.AttributeType.TEXT);
		GroupResourceViewAttribute name = new GroupResourceViewAttribute(nameAttribute, getViewTitleAttributes());

		categories.add(new Category("Office", getOfficeResourceViewAttributes()));
		categories.add(new Category("Contact", getContactResourceViewAttributes()));
		categories.add(new Category("Charges", getChangeResourceViewAttributes()));

		detailView = new DetailView("Lecturer", name, categories);
		detailView.setImage(getImage());
	} catch (DisplayViewException ex) {
		detailView = null;
	}
...
private static List<ResourceViewAttribute> getOfficeResourceViewAttributes() {
	List<ResourceViewAttribute> officeAttributes = new ArrayList<>();

	DisplayViewAttribute addressAttribute = new DisplayViewAttribute("address", ViewAttribute.AttributeType.LOCATION);
	addressAttribute.setAttributeLabel("Address");
	addressAttribute.setClickActionAndroid(true);
	SingleResourceViewAttribute address = new SingleResourceViewAttribute(addressAttribute);
	officeAttributes.add(address);

	DisplayViewAttribute roomAttribute = new DisplayViewAttribute("roomNumber", ViewAttribute.AttributeType.TEXT);
	roomAttribute.setAttributeLabel("Room");
	roomAttribute.setClickActionAndroid(true);
	SingleResourceViewAttribute room = new SingleResourceViewAttribute(roomAttribute);
	officeAttributes.add(room);

	return officeAttributes;
}
...
\end{lstlisting}

\newpage

\begin{lstlisting}[label=lst:inputview_impl,
language=java,
firstnumber=1,
caption=Erstellung einer InputView.]				   
...
List<InputViewAttribute> inputViewAttributes = new ArrayList<>();

InputViewAttribute title = new InputViewAttribute("title", ViewAttribute.AttributeType.TEXT, "Title", "Title is missing!");
title.setAttributeLabel("Title");
inputViewAttributes.add(title);

InputViewAttribute firstName = new InputViewAttribute("firstName", ViewAttribute.AttributeType.TEXT, "FirstName",
"Firstname is missing!");
firstName.setAttributeLabel("Firstname");
inputViewAttributes.add(firstName);

InputViewAttribute lastName = new InputViewAttribute("lastName", ViewAttribute.AttributeType.TEXT, "Lastname",
"LastName is missing!");
lastName.setAttributeLabel("Lastname");
inputViewAttributes.add(lastName);

InputViewAttribute mail = new InputViewAttribute("email", ViewAttribute.AttributeType.MAIL, "E-Mail", "E-Mail is missing!");
mail.setAttributeLabel("E-Mail");
inputViewAttributes.add(mail);

InputViewAttribute phone = new InputViewAttribute("phone", ViewAttribute.AttributeType.PHONE_NUMBER, "Phone Number",
"Phone number is missing!");
phone.setAttributeLabel("Phone Number");
inputViewAttributes.add(phone);

InputViewAttribute address = new InputViewAttribute("address", ViewAttribute.AttributeType.TEXT, "Address", "Address is missing!");
address.setAttributeLabel("Address");
inputViewAttributes.add(address);

InputViewAttribute room = new InputViewAttribute("roomNumber", ViewAttribute.AttributeType.TEXT, "Room", "Room is missing!");
room.setAttributeLabel("Room");
inputViewAttributes.add(room);

InputViewAttribute weLearn = new InputViewAttribute("homepage", ViewAttribute.AttributeType.URL, "welearn",
"welearn URL is missing!");
weLearn.setAttributeLabel("welearn");
inputViewAttributes.add(weLearn);

InputView inputView;
try {
	inputView = new InputView("Lecturer", inputViewAttributes);
} catch (InputViewException ex) {
	inputView = null;
}
...
\end{lstlisting}

\newpage

\begin{lstlisting}[label=lst:cardview_impl,
language=java,
firstnumber=1,
caption=Erstellung einer CardView.]			
...
List<ResourceViewAttribute> resourceViewAttributes = new ArrayList<>();

DisplayViewAttribute titleAttributes = new DisplayViewAttribute("title", ViewAttribute.AttributeType.TEXT);
titleAttributes.setAttributeLabel("Title");
SingleResourceViewAttribute title = new SingleResourceViewAttribute(titleAttributes);
resourceViewAttributes.add(title);

DisplayViewAttribute nameAttribute = new DisplayViewAttribute("name", ViewAttribute.AttributeType.TEXT);
nameAttribute.setFontSize(DisplayViewAttribute.FontSize.LARGE);
List<DisplayViewAttribute> nameAttributes = new ArrayList<>();

DisplayViewAttribute firstNameAttributes = new DisplayViewAttribute("firstName", ViewAttribute.AttributeType.TEXT);
firstNameAttributes.setAttributeLabel("FirstName");
nameAttributes.add(firstNameAttributes);

DisplayViewAttribute lastNameAttributes = new DisplayViewAttribute("lastName", ViewAttribute.AttributeType.TEXT);
lastNameAttributes.setAttributeLabel("LastName");
nameAttributes.add(lastNameAttributes);

GroupResourceViewAttribute name;
try {
	nameAttribute.setFontColor("#000");
	name = new GroupResourceViewAttribute(nameAttribute, nameAttributes);
} catch (DisplayViewException ex) {
	name = null;
}
resourceViewAttributes.add(name);

DisplayViewAttribute mailAttribute = new DisplayViewAttribute("email", ViewAttribute.AttributeType.MAIL);
mailAttribute.setAttributeLabel("E-Mail");
mailAttribute.setClickActionAndroid(true);
SingleResourceViewAttribute mail = new SingleResourceViewAttribute(mailAttribute);
resourceViewAttributes.add(mail);

DisplayViewAttribute phoneAttribute = new DisplayViewAttribute("phone", ViewAttribute.AttributeType.PHONE_NUMBER);
phoneAttribute.setAttributeLabel("Phone Number");
phoneAttribute.setClickActionAndroid(true);
SingleResourceViewAttribute phone = new SingleResourceViewAttribute(phoneAttribute);
resourceViewAttributes.add(phone);

DisplayViewAttribute addressAttribute = new DisplayViewAttribute("address", ViewAttribute.AttributeType.LOCATION);
addressAttribute.setAttributeLabel("Address");
addressAttribute.setClickActionAndroid(true);
SingleResourceViewAttribute address = new SingleResourceViewAttribute(addressAttribute);
resourceViewAttributes.add(address);

DisplayViewAttribute roomAttribute = new DisplayViewAttribute("roomNumber", ViewAttribute.AttributeType.HOME);
roomAttribute.setAttributeLabel("Room");
roomAttribute.setClickActionAndroid(true);
SingleResourceViewAttribute room = new SingleResourceViewAttribute(roomAttribute);
resourceViewAttributes.add(room);

DisplayViewAttribute welearnAttribute = new DisplayViewAttribute("homepage", ViewAttribute.AttributeType.URL);
welearnAttribute.setAttributeLabel("welearn");
welearnAttribute.setClickActionAndroid(true);
welearnAttribute.setLinkDescription("welearn");
SingleResourceViewAttribute welearn = new SingleResourceViewAttribute(welearnAttribute);
resourceViewAttributes.add(welearn);

DisplayViewAttribute imageAttribute = new DisplayViewAttribute("profileImageUrl", ViewAttribute.AttributeType.PICTURE);
imageAttribute.setAttributeLabel("ProfileImage");
imageAttribute.setPicturePosition(DisplayViewAttribute.PicturePosition.LEFT);
SingleResourceViewAttribute image = new SingleResourceViewAttribute(imageAttribute);
resourceViewAttributes.add(image);

CardView cardView;

try {
	cardView = new CardView("Lecturer", resourceViewAttributes, name);
} catch (DisplayViewException ex) {
	cardView = null;
}
...
\end{lstlisting}

\end{document}


